%!TEX root = ../head.tex

\chapter{Vorlesung}
\section{Einführung}

\subsection{Big Data}
"Big Data hat die Chance die geistige Mittelschicht in Hartz IV zu bringen"

\section{Vorlesung}
\subsection{ZIH}
\begin{itemize}
	\item HAEC
	\item CRESTA Perfomrance optimization
	\item MPI correctness checking: MUST
	\item Architecture of the new system (HRSK-II)
\end{itemize}
\subsection{Begriffe und Definitionen}
\begin{itemize}
	\item Der Begriff Rechnerarchitektur wurde von dem englischsprachigen Begriff computer architecture abgeleitet
	\item Computer architecture ist eine Teildisziplin des Wissenschaftsgebietes computer enginering, welches die überwiegend ingeniermäßige Herangehensweise beim Entwurf und der Optimierung von Rechnersystemen deutlich zum Ausdruck bringt.
	\item Zwei Deutungen des englischen Begriffs "Architecture"
	\item Zur Definition der Rechnerarchitektur
	\begin{itemize}
		\item Architektur: Ausdruck insbesondere der Möglichkeiten der Programmierschnittstelle
		\begin{itemize}
			\item Maschinenbefehlssatz
			\item Registerstruktur
			\item Adressierungsmodi
			\item Unterbrechungsbehandlung
			\item Ein- und Ausgabe-Funktionalität
		\end{itemize}
		\item Komponenten / Begriffsbildung
		\begin{itemize}
			\item Hardwarestruktur
			\item Informationsstruktur (Maschinendatentypen)
			\item Steuerungsstruktur
			\item Operationsprinzip
		\end{itemize}
	\end{itemize}
	\item Taxonomie
	\item Dreiphasenmodell zum Entwurf eins REchnersystems
	\begin{itemize}
		\item Bottom-up (Realisierung \(\to\) Implementierung \(\to\) Rechnerarchitektur)
		\item Top-down (Rechnerarchitektur \(\to\) Implementierung \(\to\) Realisierung)
		\item Rückwirkungen durch den technologischen Stand						\end{itemize}
\end{itemize}




