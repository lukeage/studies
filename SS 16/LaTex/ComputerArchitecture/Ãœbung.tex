%!TEX root = ../head.tex

\chapter{Übung}
\section{Einführung}
\subsection{von-Neumann}
\subsubsection{Komponenten des v. Neumann Architektur}
	\begin{itemize}
		\item CPU
		\begin{itemize}
			\item Steuerwerk
			\begin{itemize}
				\item steuert die Befehlsabarbeitung
				\item Befehlszähler (program counter)\(\to\) Instruction Fetch (Befehl holen) \\ \(\to\) Adresse des Befehls steht im pc		
				\item Befehlsregister \(\to\) Befehlswort ins Befehlsregister laden
				\item Befehlsdekoder \(\to\) ID (Instruktction Decode)
				\item zentrale Steuerschleife \(\to\) EX (execute - Befehl ausführen)
				\begin{itemize}
					\item CISC, Abarbeitung des Befehls unter Aufsicht der zentralen Steuerschleife
					\item RISC \(\to\) nutzt das Rechenwerk, ALU
				\end{itemize}
				\item Steuer -und Statusregister (Flag Overflow) \(\to\) WB (Write Back)
			\end{itemize}
			\item Rechenwerk
		\end{itemize}
		\item Speicher
		\begin{itemize}
			\item Programmkode und Daten liegen im gleichen Speicher
		\end{itemize}
		\item Bus
		\begin{itemize}
			\item v.Neumannscher-Flaschenhals: Daten + Befehle müssen über den BUS
			\begin{itemize}
				\item IF \(\to\) Bus
				\item ID
				\item EX \(\to\) Bus (wenn Operanden geholt werden
				\item WB \(\to\) Bus
			\end{itemize}
		\end{itemize}
	\end{itemize}
\subsection{v.Neumann vs. Harvard}
\subsubsection{Harvard}
\begin{itemize}
	\item Trennung von Befehls und Datenspeicher: Befehlsspeicher \(\to\) VN \(\to\) CPU \(\to\) Verbindungseinrichtung (z.B. Bus (VN)) \(\to\) Datenspeicher
	\item heutige Anwendung: Getrennter L1-Cache in L1I- und L1D-Cache
\end{itemize}
\subsection{Def. von Brooks vs Giloi}
\subsubsection{Brooks (1962)}
Rechnerarchitektur, wie andere Architekturen, ist die Kunst der Bestimmung von Nutzerbedürfnissen nach einer Struktur, die so zu entwerfen ist, dass sie jene Bedürfnisse so effektiv wie möglich hinsichtlich ökonomischer und technologischer Erfordernisse erfüllt.
\begin{itemize}
	\item gilt auch für jede Bauarchitektur
	\item bis Ende der 70er Jahre bezog sich Rechnerarchitektur vor allem auf die Programmierschnittstelle 
	\begin{itemize}
		\item Maschinenbefehlssatz (meist Assemblerbefehle) 
		\item Interruptbehandlung (maskierbare + nichtmaskierbare Interrupts)
		\item Registersatz
		\item Adressierungsarten (Basisadressierung, indirekte Adressierung, direkte Adressierung)
		\item Ein-/Ausgabe
	\end{itemize}
\end{itemize}
\subsubsection{Giloi}
z.B. Maschinendarstellung eines Floating Point Wertes Single Precision
\begin{itemize}
	\item Single Precision \(\to\) 32 Bit
	\item IFEE 754: |Sign|Charakteristik (Exponent + Bias|Mantisse| \(\to\) Mantisse wird so weit verschoben, bis führende 1 herausfällt
\end{itemize}
\subsection{RA-Definition Begriffe}
\textbf{Rechnerarchitektur}
	\begin{itemize}
		\item Hardware-Struktur
		\begin{itemize}
			\item Hardwarebetriebsmittelstruktur
			\begin{itemize}
				\item Prozessorstruktur
				\begin{itemize}
					\item 1985 Intel 80 386 (erster 32-Bit Prozessor) \(\to\) nur Integer Unit
					\item 1987 Intel 80 387 (erster Floating Point Unit, FPU)
					\item 1993 Pentium 1: V-Pipe(IU) und U-Pipe (IU oder Teil der FPU) \(\to\) 2 Betriebsarten: IU+IU, IU+FPU
					\item 1995 Pentium Pro: P6-Architektur \(\to\) heutige Core-Architektur ist davon abgeleitet
				\end{itemize}
				\item Speicherstruktur
				\begin{itemize}
					\item intern: Register (L1,L2,L3- Cache, DRAM, Festplatte, Archiv, ...)
					\item zwischen Prozessoren: gemeinsamer Speicher \(\Rightarrow\) CPU 1 \(\to\) CPU N haben gemeinsamen MEMORY || \\ verteilter Speicher \(\Rightarrow\) CPU 1, RAM 1 \(\to\) CPU N, RAM N; verbunden durch Verbindungsnetzwerk
				\end{itemize}
			\end{itemize}
			\item Verbindungsstruktur
			\begin{itemize}
				\item intern
				\begin{itemize}
					\item Adressbus
					\item Steuerbus
					\item Datenbus
				\end{itemize}
				\item extern 
				\begin{itemize}
					\item Verbindungsnetzwerk unterschiedlicher Typologie (Hypercube, 2D Gitter, ...)
				\end{itemize}
			\end{itemize}
			\item Kooperationsregeln (z.B. Master-Slave)
		\end{itemize}
		\item Operationspprinzip
		\begin{itemize}
			\item Informationsstruktur
			\begin{itemize}
				\item Klassen von Datentypen (Byte, Wort, ...)
				\item Menge der Maschinendarstellungen der Datenobjekte
			\end{itemize}
			\item Steuerungsstruktur
			\begin{itemize}
				\item Ablaufsteuerung, pc-getrieben \(\to\) unsere üblichen Rechner 
				\item Ablaufsteuerung, datengetrieben (Datenflussrechner) \(\to\) wenn die Daten da sind wird automatisch die Operation ausgeführt
				\item Datenzugriffssteuerung  \(\to\) Zugriff über Adresslogik, einfache Wertzuordnung, Assoziativer Zugriff (Adresse und Inhalt werden gemeinsam gespeichert) \(\to\) Caches
			\end{itemize}
		\end{itemize}
	\end{itemize}
\subsubsection{Begriffsklärung}
\begin{itemize}
	\item [RA]/[HW-Struktur]/[HW-Betriebsmittel-Struktur]/[Prozessorstruktur]/[\textbf{Steuerwerk}]
	\item [RA]/[HW-Struktur]/[HW-Betriebsmittel-Struktur]/[Speicherstruktur]/[\textbf{Register}]
	\item [RA]/[HW-Struktur]/[Verbindungsstruktur]/[\textbf{Speicherbus}]
	\item [RA]/[Operationsprinzip]/[Informationsstruktur]/[Menge der Maschinendarstellungen der Datenobjekte]/[\textbf{Festkommadatenformat} nach IEEE 754]
	\item[RA]/[Operationsprinzip]/[Informationsstruktur]/[Klassen von Datentypen]/\\ Strukturdatentypen]/ [\textbf{Doppelt verkettete Liste}]
	\item [RA]/[Hardware-Struktur]/[HW-Betriebsmittelstruktur]/[Speicherstruktur]/[\textbf{Cache}]
	\item [RA]/[Operationsprinzip]/[Informationsstruktur]/[Menge der Maschinendarstellungen der Datenobjekte]/[\textbf{Gleitkomma-Datenformate} nach IEEE 754]
	\item [RA]/[Operationsprinzip]/[Steuerungsstruktur]/Ablaufsteuerung]/\\\textbf{Program-Counter getriebene Ablaufsteuerung}]
	\item [RA]/[Operationsprinzip]/[Informationsstruktur]/[Menge der Funktionen, die auf die Datenobjekte anwendbar sind]/[\textbf{Assemblerbefehl: ADD R6, R4, R1}]
	\item [RA]/[HW-Struktur]/[\textbf{Verbindungsnetzwerk zwischen den Prozessoren}]
	\item [RA]/[Operationsprinzip]/[Steuerungsstruktur]/Dateizugriffsteuerung]/[\textbf{Zugriff auf den Cache}]
\end{itemize}
\section{Einführung}
\subsection{Moores Law}
Die Anzahl der Transistoren pro Chip verdoppelt sich alle 1,5 bis 2 Jahre (1965).
\subsubsection{Was macht man mit diesen Transistoren?}
\begin{itemize}
	\item Erhöhung der Prozessorleistung \\ \(\to\) Taktfrequenz kann erhöht werden, da bei kleineren Transistoren kleinere Kapazitäten umgeladen werden müssen (CMOS-Technik)
	\begin{itemize}
		\item 1985 Intel 80382 (erster 32-Bit Prozessor) \\ \(\to\) hatte nur eine Integer-Unit und keinen Cache
	\end{itemize}
	\item Verarbeitungsbreite erhöht (4-bit, 8-bit, 16-bit, 32-bit, 64-bit)
	\item Erhöhung der Anzahl der Verarbeitungseinheiten
	\begin{itemize}
		\item 1989: 80486
		\begin{itemize}
			\item eine FPU (Floating Point Unit)
			\item eine IU (Integer Unit)
		\end{itemize}
		\item Intel Itanium
		\begin{itemize}
			\item 2 FPU's
			\item 6 IU's
		\end{itemize}
	\end{itemize}
\end{itemize}
\subsubsection{Weshalb wir die Anzahl der Verarbeitungseinheiten nicht weiter erhöht?}
ILP (Instruction Level Parallelism)\\
Problem: Zu viele Funktionseinheiten führen zu keinem zusätzlichen Geschwindigkeitsgewinn (Sp. Speedup), da viele Befehle Datenabhängigkeiten aufweisen und so zu wenig unabhängige Befehle vorhanden sind.
\subsubsection{Multicore-Prozessoren}
\begin{itemize}
	\item [2001] IBM: erster DualCore Prozessor
	\item [2005] Intel, AMD 	
\end{itemize}
Problem: Wenn man Leistung umsetzen will muss man sich mit paralleler Programmierung beschäftigen.
\subsubsection{Lücke zwischen Prozessorleistung und Zugriffszeit auf den DRAM verkleinern}
\(\to\) Caches: benötigen ca. \(\frac{1}{3}\) der Chipfläche
\begin{itemize}
	\item L1-Cache: Harvard-Architektur
	\begin{itemize}
		\item L1-Instruction-Cache
		\item L1-Data-Cache
	\end{itemize}
	\item L2-Cache: meistens nicht getrennt (es gibt Ausnahmen: Intel Itanium Motecito)
	\item L3-Cache gemeinsam
\end{itemize}
\subsubsection{Integration weiterer Baugruppen}
z.B.: GPU (Graphic Processing Unit)
\subsection{Klassifikationen nach Flynn}
\begin{enumerate}
	\item Flynnsche Kategorien
	\begin{enumerate}
		\item SISD Single Instruction Stream Single Data Stream
		\begin{itemize}
			\item klassischer von Neumann Rechner
			\item Single Core Processor unter bestimmten Voraussetzungen:
			\begin{itemize}
				\item Eingang: sequentielle Folge von Befehlen
				\item Ausgang: Sequentielle Folge von Ergebnissen
			\end{itemize}
		\end{itemize}
		\item SIMD Single Instruction Stream Multiple Data Streams
		\begin{itemize}
			\item Vektoraddition kann man durchführen mit
			\begin{itemize}
				\item Vektorrechner
				\item Feldrechner: ein Universalprozessor steuert die gesamte Abarbeitung, sehr viele einfache Verarbeitungseinheiten (processing elements, PE) führen zu einem Zeitpunkt die gleiche Operation aus.
				\item Unterschiede:
			\end{itemize}
		\end{itemize}
		\item MISD Multiple Instruction Streams Single Data Stream\\\(\to\) leere Klasse
		\item MIMD Multiple Instruction Streams Multiple Data Streams
		\begin{itemize}
			\item Multiprozessorsystem (MPS)
			\item Cluster von Workstations 
			\item Nachteil: viel zu grob für alle MPS und COW \\ \(\to\)  in der Literatur gibt es Erweiterungen zur Flynnschen Klasse MMID:
		\end{itemize}
	\end{enumerate}
		
	\medskip
						
	\Tree [.MIMD [.{Speichergekoppelte MPS} [.NUMA NL-NUMA CL-NUMA ] UMA COMA ] [.{Nachrichtengekoppelte MPS} COW MPP ] ].MIMD
	
	\item Flynssches Klassifikationsschema
\end{enumerate}
