%!TEX root = ../head.tex

\chapter{Vorlesung}
\section{Einführung}

Gründe für DBS-Einsatz:
\begin{itemize}
\item Effizienz und Skalierbarkeit
\item Fehlerbehandlung und Fehlertoleranz
\item Mehrbenutzersynchronisation
\end{itemize}

ANSI - Database
\begin{itemize}
\item Standard siehe 1VL
\end{itemize}

Geschichte der Datenbanktechnologie
\begin{itemize}
\item siehe 1VL(28 ff.)
\end{itemize}


Databases vs Information Retrieval
\begin{itemize}
\item Information Retrieval 1VL(44)
\begin{itemize}
\item Suche nach Dokumenten
\item Nimmt ständig zu
\item In welchem Datenbstadn wird gesucht? etc...
\end{itemize}
\end{itemize}

Databases vs Big Data
\begin{itemize}
\item Big Data 1VL(47)
\end{itemize}

\section{Konzeptueller Entwurf}
\subsection{Drei Phasen des Datenbank-Entwurfs (4, ff.)}
\subsubsection{Phasen der SW-Entwicklung}
\begin{itemize}
	\item Anforderungs-analyse \(\to\) Vorstudie
	\item Fachentwurf \(\to\) Fachknozept
	\item IT-Entwurf \(\to\) IT-Konzept
	\item Implementierung \(\to\) Module/Klassen/DB-Tabellen
\end{itemize}
\subsubsection{Phasen des DB-Entwurfs}
\begin{itemize}
	\item nach Fachentwurf: fachliche Anforderungen an Datenstrukturen \(\to\) Konzeptueller DB-Entwurf \(\to\) Konzeptuelles Schema (ER-D, UML, etc.)
	\item nach IT-Entwurf: Entscheidung für logisches (Implementierungs-)Modell \(\to\) Logischer DB-Entwurf \(\to\) Logisches Schema (relational, OO, etc.)
	\item nach Implementierung: Umsetzung in konkeretem System \(\to\) Physischer DB-Entwurf \(\to\) Physisches Schema (konkretes DBS)
\end{itemize}
\begin{itemize}
	\item Datenbank = Schema + Daten
\end{itemize}
Datenbank = Schema + Daten
\subsection{Lebenszyklus einer Datenbank}
\begin{itemize}
\item Konzeptioneller Entwurf (12)
\item Logischer Entwurf (13)
\item Physischer Entwurf (14)
\item Wartung, Modifikationen, Erweiterungen (14)
\item Beispiel (15)
\end{itemize}
\subsection{Prinzip eines Datenmodells (16)}
\begin{itemize}
	\item Grundlegendes Prinzip
	\item Leistung: Beschreibung
	\item Bestandteile
	\item Skizze (17)
\end{itemize}
\subsection{Entity-Relationship-Modell}
\subsubsection{Entitäten (20)}
	\begin{itemize}
		\item Definition
		\begin{itemize}
			\item Existiert in der realen Welt, unterscheidet sich von anderen Entitäten
			\item Eine Entität ist ein Objekt der realen oder der Vorstellungswelt, über das Informationen gespeichert werden sollen
			\item Es ist im Sinne der Anwendung eindeutig berschreibbar und von anderen unterscheidbar
			\item Gleichartige Entitäten werden zu Entitätstypen (Entitätsmengen) zusammengefasst
		\end{itemize}
		\item Anmerkung
		\begin{itemize}
			\item Welche Entitäten zusammengehören, ist von Semantik der Anwendung abhängig
		\end{itemize}
		\item Merkmale von Entitätstypen (21)
		\begin{itemize}
			\item Nur für die Anwendung relevante Merkmale werden modelliert
			\item Beschreiben eine charakteristische Eigenschaft eines Eintitätstypes
			\item Werte eines Attributes aus Wertebereichen wie INTEGER, REAL, STRING
		\end{itemize}
		\item Schlüsselattribut(e)
		\begin{itemize}
			\item Ein Attribut oder eine Menge von Attributen, anhand deren Entitäten eines Entitätstyps unterscheiden lassen
			\item Werden durch Unterstreichung gekennzeichnet
			\item Beispiel: die ISBN-Nummer identifiziert das Buch
		\end{itemize}
		\end{itemize}
\subsubsection{Beziehungen / Relationships (22)}
	\begin{itemize}
		\item Abbildung von Zusammenhängen zwischen Entitäten
		\item Homogene Menge von Beziehungen wird zu Beziehungstyp zusammengefasst
		\item binär / n-när
		\item Kardinalitäten Titel \(\leftrightarrow\) Exemplar
		\item Bemerkungen
		\begin{itemize}
		\item Ein Entitätstyp darf in einem Beziehungstyp mehrfach vorkommen
		\item Mehr als zweistellige Beziehungstypen dürfen vorkommen
		\item Beziehungstypen können auch Attribute besitzen
		\end{itemize}
	\end{itemize}
\subsubsection{Beispiel eines ER-Diagramms (23)}
\subsubsection{Beispiel Funtkionalitäten (24)}
\subsubsection{Funktionalität von Beziehungstypen (25)}
	\begin{itemize}
		\item Beispiele (26 ff.)
	\end{itemize}
\subsubsection{Besonderheiten (32 ff.)}
	\begin{itemize}
		\item Rolle
		\begin{itemize}
			\item Anfrage an DB: "Gib mir alle Angestellten, die mehr verdienen als ihr Chef"
		\end{itemize}
		\item Extended-ER
		\begin{itemize}
			\item Weak Enitites
			\begin{itemize}
				\item ID nur im Kontext eindeutig (Bsp.: Stuhlnummer in Hörsaal 003 \(\leftrightarrow\) Stuhlnummer in Hörsaal 004)
			\end{itemize}
			\item Strukturierte Attribute
			\begin{itemize}
				\item Min-Max Beziehung (35 ff.)
			\end{itemize}
			
		\end{itemize}
	\end{itemize}
\subsubsection{Entwurf eines ER Diagramms (38 ff.)}
\subsubsection{Varianten für mehrstellige Beziehungstypen (40)}

