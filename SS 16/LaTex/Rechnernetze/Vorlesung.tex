% !TEX root = ../../Head.tex


\chapter{Vorlesung}

\section{Einführung}
\begin{itemize}
	\item Anwendungsfelder Rechnernetze (1.4)
	\begin{itemize}
		\item Geschäftsanwendungen - gemeinsame Nutzung von 							Resourcen
		\item Privatbereich - Informationszugriff (z.B. WWW, IM)
		\item Mobile Benutzer - Textnachrichten, ...
		\item Gesellschaftliche Aspekte - Copyright, Profile, ...
	\end{itemize}
	\item Client Server Modell (1.5)
	\item Peer-to-Peer Communication (1.6)
	\item Basis-Netzstruktur (1.7)
	\begin{itemize}
		\item Übertragungsmodi
		\begin{itemize}	
			\item Verbindungsorientiert
			\item Verbindungslos (z.B. IP)
			\item Leitungsvermittelt
			\item Paketvermittelt (flexibler, 												ressourcenschonend)
		\end{itemize}
	\end{itemize}
	\item Schichtenarchitektur - ISO/OSI Referenzmodell (1.8)
	\begin{itemize}
		\item International Organization for Standardization
		\item Open Systems Interconnection
		\item Schichtenübersicht auf 1.8 ff.
	\end{itemize}
	\item Integriertes Referenzmodell (Tanenbaum) (1.11)
	\begin{itemize}
		\item Protokollimplementierung oft abweichend vom Referenzmodell
	\end{itemize}
	\item Besipiel Datenübertragung (1.12)
	\item Schichteneffizienz (1.13)
	\item Dienste - Begriffsklärung (1.14)
	\begin{itemize}
		\item Beispiel Ablaufdiagramm (1.15)
	\end{itemize}
	\item Netzkopplung - Basis-Topologien
	\begin{itemize}
		\item Punkt-zu-Punkt-Kanäle (Unicast)
		\item Rundsendekanäle (Broadcast)
		\item Klassifizierung nach Ausdehnung (1.17)
		\begin{itemize}
			\item Pan - Personal Area Network
			\item LAN - Local Area Network
			\item MAN - Metropolitan Aria Network
			\item WAN - Wide Area Network (1.18)
		\end{itemize}
		\item Mobilität || Leistung (1.19)
		\item Konzepte - Layer-N-Gateway(1.20)
		\item Beispiel (1.21)
	\end{itemize}
	\item Internet(1.22 ff)
	\begin{itemize}
		\item Internet
		\begin{itemize}
			\item Geschichte des Internet (1.24 ff)
			\item Normen (1.26)
		\end{itemize}
		\item Intranet (1.22)
	\end{itemize}
\end{itemize}

\section{Bitübertragungsschicht}

\subsection{Nachrichtentechnische Kanäle}
\begin{itemize}
	\item Aufgabe: Physikalische Bitübertragung mittels Transformation in elektromagnetisches Signal
	\item Daten \(\to\) Kanal \(\rightsquigarrow\) Störeinflüsse \(\to\) Daten

\subsubsection{Kenngrößen (2.4 ff)}
	\begin{itemize}
		\item Bandbreite B: Breite des Frequenzbereichs eines Kanals, in dem ohne größere Dämpfung übertragen wird
		\item Baudrate %\( \text{BR} = \fraq{\text{Signalschritte}}{\textsc{s}} \)
		\item Bitrate
		\item Nyquist Theorrem \( b < 2 \cdot B \cdot ld(S) \)
		\begin{itemize}
			\item Erweiterung durch Shannon \( b < B \cdot ld(1+SNR)\)
			\item Kombination \(b < \text{min}(2 \cdot B \cdot ld(S)\text{ ; } B \cdot ld(1+SNR) )\)
		\end{itemize}		
	\end{itemize}

\subsubsection{Leitungscodes}
	\begin{itemize}
		\item Wie soll Folge von 0en und 1en übertragen werden?
		\item NRZ: \textbf{N}on-\textbf{R}eturn-to-\textbf{Z}ero  (2.6)
		\item Manchester-Codierung
		\item NRZI: NRZ-\textbf{I}nverted (2.7)
		\begin{itemize}
			\item Signaländerung bei 1, keine Signaländerung bei 0
			\item Vortei: hohe Netto-Datenrate
			\item Nachteil: Probleme bei langer Folge von Nullen
			\item Lösung: 4B/5B Code
			\begin{itemize}
				\item jeweils 4 Bits Daten werden auf 5-Bit-Muster abgebildet \(\to\) 25% Overhead (statt 100% wie bei Manchester
				\item durch 4B/5B-Code treten niemals mehr als 3 Nullen nacheinander auf
			\end{itemize}
		\end{itemize}
	\end{itemize}
\end{itemize}

\subsection{Übertragungsmedien}
\subsubsection{Elektrische Leitungen}
\begin{itemize}
	\item Twisted Pair (2.8)
	\begin{itemize}
		\item isolierte Kupferdräthe von 0,4 bis 1mm Stärke
		\item Paarweise verdrillt \(\to\) Reduzierung von Störungen
		\item Üblicherweise 4 Paare pro Kabel
		\item Mehrere Kilometer Reichweite, mehrere MBit/s, preiswert
		\item Signal aus Spannungsdifferenz zwischen den 2 Kabeln übertragen
		\item Cat 3
		\item Cat 5
		\item Cat 6
		\item Cat 7
	\end{itemize}
	\item Koaxialkabel (2.9)
	\begin{itemize}
		\item mehrere km, mehrere MBit/s, T-stecker ode rTap
		\item 50-Ohm-Kabel: für digitale Übertragung
		\item 75-Ohm-Kabel: für analoge Übertragung und Kabelfernsehen
		\item Kabelfernsehen \(\to\) Breitband-Koaxialkabel, häufig mit analoger Übertragung bis ca. 1 GHz, bidirektionaler Ausbau für Internet-Zugang via Kabel
	\end{itemize}
\end{itemize}

\subsubsection{Optische Leitungen und Sichtverbindung}
\begin{itemize}
	\item Optische Leitungen
	\begin{itemize}
		\item Lichtwellenleiter (LWL) / "Glasfaser"
		\begin{itemize}
			\item bis TBit/s-Bereich, über viele km Entfernung
			\item Monomodefaser: nur eine ausbreitungsfähige Wellenform
			\item Multimodefaser: verschiedene ausbreitungsfähige Wellenformen
			\item Gradientenfaser: schrittweise Änderung des Brechungsindex
		\end{itemize}
	\end{itemize}
\end{itemize}

\subsubsection{Sichtverbindung}
\begin{itemize}
	\item Infrarotverbindung
	\item Richtfunkstrecken
\end{itemize}

\subsubsection{Satelliten / Zellularfunk (2.11)}
\begin{itemize}
	\item Satelliten
	\begin{itemize}
		\item Getrennte Aufwärts-/Abwärtsbänder
		\item Bandbreite von 500MHz, z.B. in mehrere 50 MBit/s - Kanäle oder 800 digitale Sprachkanäle mit 64 kBit/s
		\item Zuordnung kurzer Zeitabschnitte zu einzelnen Kanälen (Zeitmultiplex)
		\item Lange Laufzeiten (ca. 250 bis 300ms)
	\end{itemize}
	\item Zellularfunk
	\begin{itemize}
		\item Aufteilung eines geographischen Bereichs in Funkzellen mit spezifischen Frequenzbändern
		\item Beispiel: GSM (Global System for Mobile Communication)
	\end{itemize}
\end{itemize}

\subsubsection{Strukturierte Verkabelung (2.12}
\begin{itemize}
	\item Ziel: Systematische, gut wartbare und erweiterbare Kabelinfrastruktur
	\item Trennung in drei wesentliche Bereiche (jeweils sternförmig hierarchisch)
	\begin{itemize}
		\item Primärebene
		\item Sekundärebene
		\item Tertiärebene
	\end{itemize}
\end{itemize}

\subsection{Mehrfachnutzung von Kanälen}
\subsubsection{Frequenzmultiplex (2.13)}
\begin{itemize}
	\item getrennte Frequenzbänder (mit z.B. 3000 Hz) und zwischengeschaltete Sperrbänder (mit z.B. 500 Hz)
\end{itemize}

\subsubsection{Orthogonales Frequenzmultiplex (Orthogonal FDM, OFDM)}
\begin{itemize}
	\item Überlaguerung der Kanäle ohne Sperrbänder \(\to\) effizienter
	\item Empfänger: Trennung der Signale mehrerer Bänder durch schnelle Fouriertransformation
	\item Einsatz: Wlan, Kabelnetze, 4G Mobilfunk, LTE, ...
\end{itemize}

\subsubsection{Zeitmultiplex (2.14)}
\begin{itemize}
	\item Zyklische Kanalzuteilung
\end{itemize}

\subsubsection{Statistisches Zeitmultiplex}
\begin{itemize}
	\item flexible Zuteilung nach Bedarf
\end{itemize}

\subsubsection{Codemultiplex (CDM, 2.15)}
\begin{itemize}
	\item Didizierte (Kodierungs-)Codes pro Teilnehmerpaar
\end{itemize}

\subsubsection{Wellenlängenmultiplex (WDM)}
\begin{itemize}
	\item Variation von Frequenzmultiplex, indem direkte optische Einkopplung mehrerer \\Lichtwellenleiter (mit Licht unterschiedlicher Wellenlängen) in einen besonders leistungsfähigen Lichtwellenleiter erfolgt
	\item entsprechende Wiederauskopplung im Zielsystem
\end{itemize}
\subsection{Datenübertragung}
\subsubsection{Signalklassen (2.16)}
\begin{itemize}
	\item Wert/Zeit kontinuierlich \(\leftrightarrow \) Wert/Zeit diskret
	\item Beispiele (2.17)
	\begin{itemize}
		\item Wert- und zeitkontinuierlich: analoges Telefon
		\item Wertkontinuierlich, zeitdiskret: Prozesssteuerung mit periodischen Messpunkten
		\item Wertdiskret, zeitkontinuerlich: digitale Temperaturanzeige
		\item Wert- und zeitdiskret: digitale Übertragung mit isochronem Taktmuster; z.B. Sprachübertragung über digitale Kanäle
	\end{itemize}
\end{itemize}
\subsubsection{Beispiel: Telefonsystem (2.18)}
\subsubsection{Sprachübertragung über digitale Kanäle (2.19)}
\begin{itemize}
	\item Analoge Eingangssignale (Sprache) vor Übertragung im Kernnetz zu digitalisieren: Codec (Coder-Decoder)
	\item Basis: Abtasttheorem nach Shannon \( f_(A) > 2\cdot f_(G) \)
	\item PCM: Pulse Code Modulation
	\begin{itemize}
		\item Bsp.: Grenzfrequenz (Telefon) : 3400 z; Abtastfrequenz: 8000 Hz
		\item logarithmische Quantisierungsintervalle \(\to\) Quantisierungsfehler begrenzen
	\end{itemize}
\end{itemize}
\subsubsection{Datenübertragung über analoge Kanäle}
\begin{itemize}
	\item Modem: Übertragung digitaler Signale über analoge (2.20) Telefonverbindung
	\begin{itemize}
		\item Problem: Nicht direkt möglich wegen kapazitiver und induktiver Einflüsse
	\end{itemize}
	\item Amplitudenabtastung
	\item Periodenabtastung
	\item Phasenabtastung
	\begin{itemize}
		\item Ziel: Deutlich höhere Übertragungsleistung durch gleichzeitige Anwendung mehrerer Modulationsverfahren (2.21)
		\item Beispiele
		\begin{itemize}
			\item QPSK
			\item QAM 16
			\item QAM 64
		\end{itemize}
	\end{itemize}
\end{itemize}

\subsection{Beispieltechnologien}
\subsubsection{Digital Subscriber Line (DSL, 2.22)}
\begin{itemize}
	\item digitaler Netzzugang über herkömmliche Telefonleitungen
	\item Datenübertragung und Telefondienst gleichzeitig nutzbar
	\item Realisierung durch Nutzung höherer Frequenzbereiche
	\item hohe Datenraten, meist asymmetrisch (ADSL) bzgl. Up-/Downlink
	\item weitere Varianten:
	\begin{itemize}
		\item VDSL (Very High Bitrate) : nur über kurze Entfernungen
		\item SDSL (Symmetric): GLEICHE dATENRATE AUF Up-/Downlink
	\end{itemize}
	\item Signaltrennung (Telefon/Daten) und Modulation (basierend auf QAM, 2.23)
	\begin{itemize}
		\item CAS (Carrierless Amplitude / Pase System)
		\item DMT (Discrete Multitone)
	\end{itemize}
\end{itemize}
\subsection{Digitaler Netzzugang über Kabelmodem}
\begin{itemize}
	\item Signaltrennung zwischen Kabelfernsehen und Daten:
	\begin{itemize}
		\item Umwidmung einzelner TV-Kanäle in Datenkanäle
		\item Rückkanalfähige Verstärker erforderlich
		\item Datenraten theoretisch bis ca. 36 MBit/s, aber "Shared Medium", d.h. abhängig von der Zahl der Teilnehmer geringere Datenrate
	\end{itemize}
\end{itemize}

\section{Netztechnologie 1}
\subsection{Medienzugriff}
\subsubsection{ALOHA Protokoll}
\begin{itemize}
	\item historisches Paketfunknetz, University of Hawaii, seit 1979
	\item dezentrale Stationen, Kommunikation über Zentrale
	\item unkoordiniertes Wettbewerbsverfahren (stochastisch)
	\item Kollision auf \(f_{1}\) bei Zentrale, da Senden stets möglich
	\item Fehlerbehandlung durch Wiederholung, falls nach Zeit \(t\) keine Quittung auf \(f_{2}\)
	\item kein Mithören während des Sendevorgangs
\end{itemize}
\subsubsection{ALOHA Beispiel}
\begin{itemize}
	\item Pure ALOHA: Max. etwa 18 Prozent des Kanaldurchsatzes
	\item Slotted ALOHA: Max. etwa 36 Prozent des Kanaldurchsatzes (3.6)
\end{itemize}
\subsubsection{CSMA-Verfahren}
\begin{itemize}
	\item kein Funk, sondern Coaxialkabel
	\item Abhören vor Senden (CSMA - Carrier Sense Multiple Acces)
	\item Trotzdem Kollision möglich: (1-persistent CSMA, immer sendebereit)
	\item nonpersistent CSMA: belegter Kanal wird nicht sofort erneut abgehört, erst nach zufällig verteiltem Zeitintervall; dadurch geringere Kollisionswahrscheinlichkeit
	\item p-persistent CSMA (slotted): Prüfe Kanal, sende mit Wahrscheinlichkeit p, warte sonst 1 Slot und prüfe wieder
\end{itemize}
\subsubsection{Bewertung der Verfahren (3.8)}
\subsubsection{CSMA mit "Collision Detection" (CD)}
\begin{itemize}
	\item Mithören während des Sendevorgangs
	\item Kollisionserkennung dadurch schneller möglich (ohne Warten auf Quittung)
	\item Funktioniert für ein gemeinsam genutztes Kommunikationsmedium ... (z.B. gemeinsames Kabel bei IEEE 802.3, Lukf bei IEEE 802.11, etc.)
	\item ... mit mindestens einer Station (Kollisionen mit sich selbst können erkannt werden, z.B. durch Signalreflexion am offenen Kabelende
\end{itemize}
\subsubsection{CSMA/CD-Verfahren Beispiel (3.10 ff.)}
\subsection{Ethernet}
\subsubsection{IEEE 802.3}
\begin{itemize}
	\item Zugriffsverfahren: 1-persistent CSMA/CD, in Hardware auf Ethernet-Karte realisiert
	\item Datenrate der Basistechnologie: 10 MBit/s
	\item Segmentlänge: 500m
	\item Kabel der Kategorie 5 oder höher bzw. Lichtwellenleiter (dann auch deutlich größere räumliche Ausdehnungen möglich)
	\item heute grundsätzlich mit Switches und Duplex-Betrieb im Einsatz
	\item dennoch Kollisionsbehandlung generell mit eingebaut:
	\begin{itemize}
		\item warte s Slots nach Kollision, s zwischen 0 und \(2^n-1\) bei n vorherigen Kollisionen zufällig gewählt
	\end{itemize}
\end{itemize}
\subsubsection{Rahmenstruktur (3.14)}
\begin{itemize}
	\item Präambel erlaubt Synchronisation mit Empfänger
	\item EtherType/Size
	\begin{itemize}
		\item \(\le 1500 \to \) Länge des Datenfelds
		\item \(\ge 1536 \to \) Typ der Daten (z.B. IP, IPv6, etc.), Länge der Daten nicht spezifiziert \(\to\) Interframe Gap als Begrenzer
	\end{itemize}
	\item Pad zum Auffüllen auf minimale Rahmenlänge wegen Kollisionsverfahren
	\item Prüfsumme: CRC (Cyclic Redundancy Check), ohne Präambel und SFD
\end{itemize}
\subsubsection{Fast- / Gigabit Ethernet}
\begin{itemize}
	\item Fast Ethernet
	\begin{itemize}
		\item 1995 als IEEE 802.3u standardisiert
		\item Datenrate 100 MBit/s
		\item Segmentlänge: 100m bei Kupferkabel, 2km bei Lichtwellenleiter
		\item Kompatibilität zu Ethernet und Cat-3-Kabel, noch CSMA/CD unterstützt aber keine Multidrop-Kabel mehr
	\end{itemize}
	\item Gigabit Ethernet
	\begin{itemize}
		\item 1999 als IEEE 802.3ab standardisiert
		\item Datenrate 1 GBit/s
		\item Vollduplex (Standard): kein CSMA/CD mehr \(\to\) keine Beschränkung der Kabellänge
		\item Halbduplex: Layer-1-Kopplung über Hub; CSMA/CD mit Modifikationen: 
		\begin{itemize}
			\item Padding - Rahmen immer auf 512 Byte auffüllen
			\item Frame-Bursting - mehrere Rahmen in einem Ethernet-Frame übertragen
		\end{itemize}
	\end{itemize}
	\item 10 / 100 GBit/s Ethernet
	\begin{itemize}
		\item für optische Verbindungen in WANs \(\to\) siehe Kapitel 4
	\end{itemize}
\end{itemize}
\subsubsection{Ethernet-Varianten für LAN (3.16)}
\subsubsection{Switched Ethernet: Beispiel (3.17)}
\begin{itemize}
	\item parallele Vermittlung aller Verkehrsströme durch Switch-Hardware
	\item Vorteil: Keine Kollisionen, jeder Station steht die volle Ethernet-Datenrate zur Verfügung \(\Rightarrow\) Ethernet wird vom "Shared Medium" zum "Switched MEdium"
	\item Aufteilung der Stationen and einem oder mehreren Switches in unterschiedliche virtuelle lokale Netze (VLAN) möglich \(\Rightarrow\) Sicherheitszonen
\end{itemize}

\subsection{Switches in der Sicherungsschicht}
Ziele (3.18):
\begin{itemize}
	\item parallele Vermittlung durch Switches, sequentiell duch Bridges (veraltet)
	\item Trennung organisatorischer Bereiche/verschieden Verkehrsströme
	\item Zuverlässigkeit und Sicherheit (gegen Störsignale und unberechtigte Weiterleitung)
	\item Begrenzung der Netzlast durch selektives Weiterleiten von Nachrichten
\end{itemize}
\subsubsection{Modell (3.19)}
\subsubsection{Transparent Bridges / Switches (3.20)}
\begin{itemize}
	\item Selbstlernend: Automatischer Aufbau von Routing-Tabellen
	\item Topologie-Erkennung durch Quelladressen, schrittweiser Tabellenaufbau
	\item Fluten, falls Zielrechner noch unbekannt
	\item Löschen von Einträgen nach bestimmter Zeit zur Anpassung an Topologieänderungen
\end{itemize}
\subsubsection{Spanning Tree (3.21)}
\begin{itemize}
	\item Problem: Mehrfachwege \(\to\) Endlosschleifen
	\item Lösung: Aufbau eines "überspannenden Baumes" mit eindeutigen Wegen durch dezentralen Algorithmus / kürzester Weg zur Wurzel
\end{itemize}
\subsubsection{Interne Realisierung (3.22)}
\begin{itemize}
	\item Parallele Vermittlung mehrerer Eingangs- an mehrere Ausgangsports
	\item Hohe Leistung, unterstützt durch Hardware-Realisierung
	\item Store-and-Forward-Switch: Gesamtes Frame wird im Switch zwischengespeichert, die Prüfsumme wird kontrolliert und erst dann wird weitergeleitet \(\Rightarrow\) einfach; Pufferung und Datenratenanpassung
	\item Cut-Through-Switch: Andommende Frames werden nach Prüfung der Zieladresse sofort weitergeleitet \(\Rightarrow\) effizienter, kürzere Verzögerung, aber problematisch bei unterschiedlichen Datenraten und bei Fehlern
\end{itemize}
\subsubsection{VLAN - Virtual Local Area Network (2.23)}
\begin{itemize}
	\item Motivation:
	\begin{itemize}
		\item Flexibilität: Änderung der Zuordnung von Geräten zu lokalen Netzen ohne neue Verkabelung
		\item Sicherheits- und Performance-Aspekte
	\end{itemize}
\end{itemize}
\subsubsection{Port-basiertes VLAN (3.24)}
\begin{itemize}
	\item Jeder Port eines Switches wird einem VLAN zugeordnet
	\item Ports können nur Mitglied eines VLANs sein
	\item redundante Links zwischen Switches benötigt
\end{itemize}
\subsubsection{Tag-basiertes VLAN - IEEE 802.1Q (3.25)}
\begin{itemize}
	\item Transport mehrerer VLAN-Pakete über einen Link \(\to\) Tagging der Pakete
	\item IEEE 802.1Q - Ergänzung des Ethernet-Headers - VLAN-Tag
	\item letzter VLAN-Fähiger Switch entfernt das VLAN-Tag wieder \(\to\) Kompatibilität
	\item VLAN Identifier = 12 Bit
	\item andere Felder (Priority und CFI) nicht für VLAN genutzt
\end{itemize}
\subsection{Drahtlose Netze für PAN und LAN (3.26)}
\subsubsection{WLAN: IEEE 802.11 (3.27)}
\subsubsection{802.11 - Medienzugriff mit CSMA/CA (3.28) ff.}
\begin{itemize}
	\item RTS/CTS - Request to Send / Clear to Send
	\item Hidden terminal: A kann C wegen begrenzter Funkreichweite nicht hören
	\begin{itemize}
		\item A sendet RTS-Signal an B, und B sendet dann CTS
		\item Alle anderen möglichen Sender (C) erhalten das CTS-Signal und stellen ihren Sendevorgang zurück
	\end{itemize}
	\item Exposed terminal (unnötiges Warten, hier durch B bei Senden nach links)
	\begin{itemize}
		\item C sendet RTS an möglichen anderen Empfänger
		\item Falls dieser beriet, erhält C das CTS und kann übertragen (unabhängig von B)
	\end{itemize}
\end{itemize}
\subsubsection{Bluetooth}
\begin{itemize}
		\item drahtlose Ad-Hoc-Piconetze (<10m), billige Ein-Chip-Lösung
		\item offener Standard: IEEE 802.15.1
		\item Einsatzgebiete:
		\begin{itemize}
			\item Verbindung von Perpheriegeräten
			\item Unterstützung von Ad-Hoc-Netzen
			\item Verbindung verschiedener Netze (z.B. drahtloses Headset mit GSM)
		\end{itemize}
		\item Frequenzband im 2,4 GHz- Bereich; Integrierte Sicherheitsverfahren \\(128-Bit-Verschlüsselung)
		\item Datenraten:
		\begin{itemize}
			\item 433,9 kBit/s asynchronous-symmetrical
			\item 723,2 kBit/s / 57,6 KBit/s asynchronous-asymmetrical
			\item 64 kBit/s synchronous, voice service
			\item Erweiterungen bis zu 20 Mbit/s (IEEE 802.15.3a: UWB (Ultra Wide Band)
		\end{itemize}
\end{itemize}
\subsubsection{ZigBee (3.30}
\subsubsection{RFID - Radio Frequency Identification (3.31)}
\begin{itemize}
	\item Klasse-1-Tags:
	\begin{itemize}
		\item bestehen aus Antenne und RFID-Chip
		\item 96-Bit-Identifikator, kleiner Speicher, passiv
		\item geringer Preis, lässt sich z.B: auf Produkte aufkleben
	\end{itemize}
	\item Lesegerät:
	\begin{itemize}
		\item aktiv, leistungsfähig, MAC-Protokolle
		\item sendet Trägersignal, wird von Tag reflektiert
	\end{itemize}
	\item Backscatter: Tag überlagert das Trägersignal mit eigenen zu sendenden Bits \(\to\) Lesegerät filtert dies Bits aus
	\item Mehrfachzugriff: modifizierte Version von Slotted ALOHA
\end{itemize}
\subsubsection{NFC - Near Field Communication}
\begin{itemize}
	\item kontaktloser Datenaustausch über Kürzeste Distanzen (4cm)
	\begin{itemize}
		\item Auflegen/anlegen des Transmitters an Lesegerät erforderlich
	\end{itemize}
	\item Datenübertragunsrate bis zu 424 kBit/s
	\item Übertragung
	\begin{itemize}
		\item verbindungslos: passive RFID-Tags
		\item verbindungsorientiert: aktive Transmitter (z.B. Smartphone)
	\end{itemize}
	\item mögliche Anwendungen
	\begin{itemize}
		\item Bezahlung per Smartphone oder Smartcard
		\item Smartphone als Türschlüssel
	\end{itemize}
	\item Kritik
	\begin{itemize}
		\item Distanz als Sicherheitsfeature ungeeignet (durch große Antennen bis zu 1m möglich) 
		\item NFC-Sicherheitsmechanismen unzureichend
	\end{itemize}
\end{itemize}

\section{Netztechnologien 2}
\subsection{Schichtenübersicht - Sicherungsschicht}
\subsection{Überblick}
\begin{itemize}
	\item Einordnung im Wesentlichen in OSI-Schichten 1 und 2
	\item PAN - Personal Area Network / LAN - Local Area Network
	\begin{itemize}
		\item Ausdehnung bis zu einigen Kilometern
		\item Privates Unternehmen / Privathaushalt als Netzbetreiber
	\end{itemize}
	\item MAN - Metropolitan Ariea Network / WAN - Wide Area Network
	\begin{itemize}
		\item weiträumige Ausdehnung, öffentlich zugänglich, dedizierte Betreiber
		\item Dienstqualität sehr wichtig (viele konkurrierrende Verkehrsströme
	\end{itemize}
	\item hier: Technologien für MAN und WAN
	\begin{itemize}
		\item MAN: Drahtloses Breitband, Token Bus/RPR, Carrier Ethernet
		\item WAN: Carrier Ethernet, MPLS, SDA, OTN (3)
	\end{itemize}
\end{itemize}
\subsection{Drahtloses Breitband}
\subsubsection{WiMAX}
\begin{itemize}
  \item Problem: Kabel-gebundenes Internet sehr teuer, vor allem in ländlichen Gegenden
  \item Lösung: drahtlose Breitbandnetze
  \item Standardisierung: IEEE 802.16 = WiMAX (Worldwide Interoperability for Microwave Acces)
  \begin{itemize}
  	\item Frequenzbereich 2 GHz - 66 GHz
  	\item erste Version 2001: stationär, Sichtverbindung nötig
  	\item heute: für mobile Anwendungen mit Fahrzeuggeschwindigkeit geeignet
  	\item 4G-Technologie mit bis zu 1 GBit/s - verbindet Aspekte von WLAN und 3G
  \end{itemize}
  \item Technologien
  \begin{itemize}
  	\item OFDM (Orthogonal Frequency-Division Multiplexing)
  	\item MIMO (Multiple Input Multiple Output)
  \end{itemize}
  \item Prinzip, grafisch (5)
  \item Bitübertragungsschicht
  \begin{itemize}
  	\item OFDMA - Orthogonal Frequency Division Multiple Acces
  	\begin{itemize}
  		\item flexible Aufteilung von OFDM-Unterträgern auf Stationen
  	\end{itemize}
  	\item Zeiitduplexverfahren TDD (Time Division Duplex)
  	\begin{itemize}
  		\item Station wechselt zwischen Senden und Empfangen auf einem Unterträger (6)
  	\end{itemize}
  	\end{itemize}
 	\item WiMax - Dienstklassen und Rahmenstruktur
  	\begin{itemize}
  		\item Konstante Datenrate (z.B. unkomprimiertes VoIP)
  		\item Echtzeit mit variabler Datenrate (z.B. kompr. VoIP)
  		\item Nichtechtzeit mit variabler Datenrate - (z.B. Übertragung großer Dateien)
  		\item Best Effort	
  	\end{itemize}
  	\item Rahmenstruktur (7)
  	\begin{itemize}
  		\item Anforderung Bandbreite \\
  		Type \(\to\) Bytes Needed \(\to\) Connection ID \(\to\) Header CRC
  		\item Datenübertragung \\
  		Type \(\to\) Flags \(\to\) Length \(\to\) Connection ID \(\to\) Header CRC \(\to\) Data \(\to\) CRC
  	\end{itemize}
\end{itemize}
\subsubsection{Steigerung der Datenrate: MIMO}
\begin{itemize}
	\item MIMO - Multiple Input Multiple Output
	\item Nutzung mehrerer Sende- und Empfangsantennen pro Gerät
	\item Signale überlagern sich, durch Mehrwege-Ausbreitung kleine Vraitionen im empfangenen Signal
	\item Kanalmatrix H wird beim Empfänger berechnet \(\to\) sehr komplex
	\item bei n Sende- und n Empfangsantennen theoretisch n-fache Kanalkapazität möglich
	\item Komplexität der Berechnung steigt überproportional mit n
	\item Nachteil MIMO: höhere Hardware-Kosten
\end{itemize}
\subsection{Token-basierte Technologien}
\subsubsection{Token Bus IEEE 802.4}
\begin{itemize}
	\item physikalischer Bus, logischer Ring (nur für Token-Weitergabe)
	\item Charakteristika: deterministisch, realzeitfähig, Prioritäten
	\item Vorläufer mit ähnlichen Eigenschaften: Token Ring (IEEE 802.5)
	\item Aufnahme neuer Stationen:
	\begin{itemize}
		\item Token-Besitzer sendet periodisch Anfragen (SOLICIT-SUCCESSOR)
		\item Station kann "Aufnahmeantrag" stellen
		\item Kollisionsbehandlung, falls sich mehrere Stationen melden
	\end{itemize}
	\item Verlassen des Ringes:
	\begin{itemize}
		\item sende SET-SUCCESSOR an Vorgänger (mit Nachfolgerangabe)
		\item Vorgänger setzt neuen Nachfolger
	\end{itemize}
	\item Ausfall einer Station:
	\begin{itemize}
		\item Vorgänger prüft nach Token-Weitergabe, b gesendet wird oder Token wiederum weitergegeben wird (ggf. mehrfach)
		\item falls nicht, sende WHO-FOLLOWS, übernächste Station wird dann durch SET-SUCCESSOR zum neuen Nachfolger
		\item Ausfall des Token-Besitzer: Neuinitialisierung
	\end{itemize}
	\item Mehrfach-Token (durch zufällige Duplizierung etc.):
	\begin{itemize}
		\item Erkennung bei unberechtigten Sendevorgängen
		\item freiwilliges Löschen des eigenen Tokens
	\end{itemize}
	\item Zahlreiche weitere Verwaltungsprotokolle
\end{itemize}
\subsubsection{RPR - Resilient Packet Ring}
\begin{itemize}
	\item RPR: Weiterentwicklung von Token Ring / Token Bus für den MAN/WAN Bereich (IEEE 802.17)
	\item Datenraten: 1 GBit/s bis 10 GBit/s
	\item doppelter Ring (Ringlets) für beide Richtungen mit optischen Verbindungen 
	\item Dienstklassen: real time, near real time, best effort
	\item Fairness: koordinierte Drosselung aller Datenströme bei Überlastung
	\item wichtigstes Feature: schnelle umschaltung auf Ersatzwege bei Leitungsunterbrechungen (in 50 ms)
	\item Spatial reuse: mehrere gleichzeitige Übertragungsvorgänge pro Ringlet möglich \(\to\) bessere Ausnutzung der Übertragungskapazität
\end{itemize}
\subsection{Carrier Ethernet}
\begin{itemize}
	\item Ethernet-Technologie hat sich auch im MAN/WAN Bereich durchgesetzt und dort andere Technologien (ATM, FDDI, Fibre Channel) verdrängt
	\item Gründe: niedrige Kosten, kompatibel zu LAN-Paketen, flexibel, einfach zu nutzen
	\item Voraussetzung: Punkt-zu-Punkt-Verbindungen (Connection-oriented Ethernet)
	\begin{itemize}
		\item kein CSMA/CD
		\item Wegfall der Längenbecshränkungen
	\end{itemize}
	\item mit Glasfaserkabeln heute Ethernet-Verbindungen von 10 bis 140 km bei 1 bis 100 GBit/s realisierbar
	\item weitere Optimierungen (siehe auch Kap. 8):
	\begin{itemize}
		\item Jumborahmen bis 9KB
		\item Bitübertragung durch 8B/10B-Codierung bzw. 64B/66B-Codierung mit geringem Overhead
	\end{itemize}
\end{itemize}
\subsubsection{Metro Ethernet Networks MEN}
\begin{itemize}
	\item Ethernet-basierte Dienste im MAN/WAN-Bereich
	\item Standardisierung: Methro Thernet Forum (MEF) und IEEE
	\item MEF-Standards enthalten Ethernet-Erweiterungen zu
	\begin{itemize}
		\item DIenste:E-line, E-LAN oder E-Tree
		\item Skalierbarkeit: inkrementell skalierbare Datenrate
		\item Zuverlässigkeit: Unterbrechungen entdecken und schnell beheben (50ms)
		\item qquality of Service: Service Level Agreements (SLAs)
		\item Service Management: Moitoring, zentrale Steuerung
	\end{itemize}
	\item verschiedene Transporttechnologien möglich: MPLS, SDH, OTN
	\item Beispiel (14)
\end{itemize}
\subsubsection{Provider Bridging}
\begin{itemize}
	\item Provider Bridges - IEEE 802.1as QinQ (Stacked VLANs)
	\item Erwetierung des Ethernet Frames:(15)
	\item Outer Tag und Destination Address werden zum Routing im Provider-Netz verwendet
	\item Weiterentwicklung: Provider Backbonde Bridges - 802.1ah Mac-in-Mac
	\item neue Ethernet-Header: (15)
\end{itemize}
\subsubsection{Dienstgüte bei Carrier Ethernet}
\begin{itemize}
	\item MEF spezifiziert Bandwidth Profiles für Ehternet Virtual Connection (EVC) = Service Level Agreement (SLA)
	\begin{itemize}
		\item Ingress Badnwidth Profile - Spezifikation für eingehenden Traffic eines User Network Interface (UNI)
		\item Egress Bandwidth Profile - Spezifikationen für ausgehenden Traffic an ein UNI
		\item Commited Information Rate (CIR) - feste zusage für kritische Daten
		\item Excess Information Rate (EIR) - weitere Kapazität für unkritische Daten
	\end{itemize}
	\item Traffic Coloring
	\begin{itemize}
		\item Grün: Service Frame liegt innerhalb der SLA = CIR
		\item Gelb: Sercie Frame liegt außerhalb der SLA, wird aber auf Best Effort Basis weitergeleitet = EIR
		\item Rot: Service Frame wird verworfen
	\end{itemize}
	\item Auch Beschränkungen für Delay, Paketverluste und Jitter können Bestandteil des SLA sein
\end{itemize}
\subsubsection{Exkurs: VPNs}
\begin{itemize}
	\item VPN = Virtual Private Network
	\item Motivation
	\begin{itemize}
		\item Anbindung entfernte Bürostandorte
		\item Einbindung mobiler Mitarbeiter
		\item Einrichten eines Forschungsnetzes (Bsp.: CERN)
	\end{itemize}
	\item Funktion 
	\begin{itemize}
		\item Einrichten dedizierter privater Kommunkationpfade ("Tunnel") durch ein öffentliches Weitverkehrsnetz
		\item meist kombiniert mit Verschlüsselung für Abhörsicherheit zischen zwei Zugangspunkten
	\end{itemize}
	\item Realisierung
	\begin{itemize}
		\item Layer 3 (Vermittlungsschicht): z.B. per IPsec - IP-basierter Tunnel zwischen 2 Endpunkten (Client und VPN-Gateway) oder 2 Routern
		\item Layer 2 (Netzwerkschicht):
		\begin{itemize}
			\item Einrichten virtueller LANs (z.B. Carrier Ethernet)
			\item Einrichten virtueller Pfade durch ein WAN (z.B. Multi Protocol Label Switching - MPLS)
		\end{itemize}
	\end{itemize}
\end{itemize}
\subsection{MPLS - Multiprotocol Label Swtitching}
\begin{itemize}
	\item Motivation
	\begin{itemize}
		\item Schnelle Weiterleitung, QoS-Garantien
		\item Entlastung von Routern
		\item Trennung von Netzwerktraffic (z.B. VPNs, QoS-Klassen)
	\end{itemize}
	\item Merkmale
	\begin{itemize}
		\item Weiterleitung von Paketen entlang vordefinierter Pfade (gesteueret druch Labels)
		\item Forward Equivalence Classes (FEC), d.h. gleiche Behandlung aller Pakete eines Datenstroms (z.B. Videokonferenz)
		\item Abbildung auf existierende Netztechnologien möglich bzw. Implementierung durch spezielle MPLS-Hardware direkt über Lichtwellenleiter
	\end{itemize}
	\item MPLS zusammen mit Carrier Ethernet
	\begin{itemize}
		\item Variante 1: MPLS-Router per Carrier Ethernet verbunden \(\to\) MPLS over Carrier Ethernet
		\item Variante 2: Carrier Ethernet als Overlay über optisches MPLS-Netz realisiert \(\to\) Carrier Ethernet over MPLS
	\end{itemize}
	\item Funktionsweise/Beispiel (19ff)
\end{itemize}
\subsubsection{MPLS-Switches / Layer-3-Switches}
\begin{itemize}
	\item Kombination von Router und Switch
	\item zunächst Routing, bei längerer Dauer von Datenströmen ggf. Druchstalten von Verbindungen auf Basis von MPLS (Multiprotocol Label Switching) 
	\item Teilweise auch Berücksichtigung von Informationen der Transportschicht ("Layer-4-Switches")
\end{itemize}
\subsubsection{MPLS Transprot Profile (MPLS-TP)}
\begin{itemize}
	\item verbindungsorientiertes MPLS, für Transprotnetze (Backbone) optimiert
	\item durch IETF standardisiert, Nachfolger von T-MPLS (ITU-Standard)
	\item vordefinierte Routen - bidirektional
	\item QoS-Erweiterungen, Netzwerkmanagement
	\item weitere Funktionen für Überwachung und Verwaltung des Netzes (in-band)
	\item Unterstützung für Netzwerkmanagement-System (NMS)
\end{itemize}
\subsubsection{MPLS - Bewertung}
\begin{itemize}
	\item gute Integration mit IP und Ethernet, einfaches Netzmanagement
	\item es wird ein eigener Header für MPLS eingeführt /zwischen IP-Header und Header des darunterliegenden Ethernet bzw. anderer Netze);
	dadurch wird Unabhängigkeit vom Basisnetz erreicht ("Multiprotocol")
	\item mehrere Anwendungen können zu einer Forward Equivalence Class zusammengefasst werden; daher flexibel und ressourcenschonend
	\item MPLS-Pfade müssen nicht explizit aufgebaut werden, sonder werden \\"on-the-fly" etabliert, z.B. durch Analyse des Datenstromes
	\item in der Praxis heute häufig (10/100) Gigabit Ethernet auf Basis von Hochleistungsswitches kombiniert mit MPLS im WAN-Bereich im Einsatz
\end{itemize}
\subsection{SONET / SDH}
\begin{itemize}
	\item SONET - Synchronous Optical Network, SDH - Syhnchronous Digital Hierarchy
	\begin{itemize}
		\item Bündelung mehrerer Datenströme über optische Leiter
		\item wird im Telefonie-Backbone welweit eingesetzt
	\end{itemize}
	\item ISDN-Telefonie: 8000 \(\cdot\) 8 Bit/Sample = 64 kBit/s
	\begin{itemize}
		\item Idee: jede Telefonverbindung belegt ein Byte pro SONET/SDH-Rahmen
		\item Rahmen alle 125 \(\mu s\) senden
	\end{itemize}
	\item SONET Basisrahmen: Block von 810 Byte = STS-1 
	\begin{itemize}
		\item 8 \(\cdot\) 810 Bit \(\cdot\) 8000 \(\cdot\) 1/s = 51,84 MBit/s
		\item 9 Zeilen und 90 Spalten - 3 Spalten Overhead, 87 Spalten Nutzdaten
	\end{itemize}
	\item höhere Datenraten durch größere Pakete (Vielfache von STS-1)
	\item hierarchisches Multiplexing/Demultiplexing
\end{itemize}
\subsection{OTN - Optical Transport Network}
\begin{itemize}
	\item Problem SONET/SDH: keine höhere Geschwindigkeit als 40 GBit/s
	\item OTN - einheitliche Übertragungstechnologie für optische Netze
	\begin{itemize}
		\item Nachfolge-Technologie von SONET/SDH
		\item kann SONET/SDH-Traffic transportieren
		\item kann auch anderen Traffic wie Ethernet, ATM oder MPLS gemsicht über eine optische Verbindung übertragen
	\end{itemize}
	\item stadardisiert durch die ITU (G.709)
	\item Datenübertragung basiert auf DWDM - Dense Wavelength Division Multiplexing
	\item unterstützt Datenraten größer als 40 GBit/s
	\item verbesserte Zuverlässigkeit
	\item Funktionen
	\begin{itemize}
		\item Forward Error Correction
		\item Frame Mapping
		\item Multiplexing (DWDM)
		\item Management von Pfaden und Performance
	\end{itemize}
	\item Beispiel (27)
\end{itemize}
\subsubsection{DWDM - Dense Wavelength Division Multiplexing}
\begin{itemize}
	\item WDM - Wavelength Division Multiplexing
	\begin{itemize}
		\item Aufteilung der Kanäle auf verschiedene Wellenlängen
		\item Hinzufügen/Trennung und Verstärkung durch rein optische Elemente möglich
		\item durch bessere Sende-Empfangstechnik können bestehende Glasfasern höhere Datenraten transprotieren \(\cdot\) Kosteneffizienz
	\end{itemize}
	\item CWDM - Coars Wavelength Division Multiplexing
	\begin{itemize}
		\item breite Trennbereiche zwischen Kanälen, ca. 5 bis 20 Kanäle pro LWL
		\item kostengünstige Sende-/Empfangstechnik
	\end{itemize}
	\item DWDM - Dense Wavelength Division Multiplexing
	\begin{itemize}
		\item kleinere Trennbereiche \(\cdot\) aufwendigere Sende-/Empfangstechnik
		\item ca. 60 - 160 Kanäle pro LWL
		\item Reichweite: ca. 80 - 160 km bei 100 Gbit/s; bis zu 1 TBit/s theoretisch möglich
	\end{itemize}
\end{itemize}

\section{Sicherungsschicht}
\subsection{Sicherungsschicht}
\subsubsection{Aufgaben}
Kommunikation zwischen Partnern im gleichen Subnetz (z.B. via Ethernet) bzw. über Punkt-zu-Punkt-Verbindung (z.B. PPP - Point-to-Point-Protocol bei Internet-Zugang oder via VPN - Virtuelles Privates Netz)
\begin{itemize}
\item Bildung von Übertragungsrahmen zur Kontrolle von Prüfsummen
\item Fehlerbehandlung
\item Flusskontrolle zur Überlastvermeidung
\item Verbindungsverwaltung
\item Interface für die Vermittlungsschicht
\end{itemize}
\subsubsection{Zuverlässigketisebenen}
\begin{itemize}
	\item Unbestätigter verbindungsloser Dienst
	\item Bestätigter verbindungsloser Dienst
	\item Bestätigter verbindungsorientierter Dienst
	\begin{itemize}
		\item Aufbau einer Verbindung vor der Übertragung
		\item nummerierte Rahmen
		\item Garantie, dass der Rahmen ankommt
		\item Reihenfolge wird beibehalten
		\item genau einmaliger Empfang möglich
	\end{itemize}
\end{itemize}
\subsubsection{Rahmenbildung}
Realisierung durch:
\begin{itemize}
	\item Bytezahl am Anfang des Rahmens
	\item Rahmenbegrenzer auf Sicherungssschicht (Flagbytes, Flagbits)
	\item Rahmenbegrenzer auf der Bitübertragungsschicht
	\item kombinierte Verfahren, z.B. Ethernet und WiFi: Präambel als Anfang des Rahmens + Bytezahl für die Länge des Rahmens
\end{itemize}
Problem bei Rahmenbegrenzern: Bitmuster des Rahmenbegrenzers muss auch im Payload zulässig sein: Byte-/Bit-Stuffing
\subsubsection{Bytezahl (6)}
\subsubsection{Bitstuffing}
Ablauf: Füge nach jeder fünften aufeinanderfolgenden 1 in den Nutzdaten jeweils eine 0 ein und entferne diese wieder beim Empfänger \(\to\) eindeutige Rahmenbegrenzung (Beispiel (7))
\subsection{Fehlerbehandlung}
Probleme: Verfälschungen durch thermisches Rauschen, Impulsstärungen, frequenzabhängige Verzerrung etc. \(\to\) Einzelbitfehler und Bündelfehler\\
Hamming Distanz d: Minimale Anzahl unterschiedlicher Bits zwischen zwei Quellcodewörtern; hier d = 3 Anzahl erkennbarer Fehler: d-1, Anzahl korrigierbarer Fehler: (d-1)/2
\subsubsection{Fehlererkennende Codes}
Paritätsicherung (9)
\subsubsection{Cydlic Redundancy Check - CRC}
Funktionsweise und Beispiel (10 ff)
\subsection{Protokolle}
\subsubsection{Stop-and-Wait-Protokoll(14)}
\subsubsection{Protokoll mit Fehlerbehandlung}
Ablauf:
\begin{itemize}
	\item Wiederhole verlorengegangene nachrichten nach Timeout, sonst sende nächsten Frame
	\item Ignoriere entstende Duplikate \(\to\) Laufnummern 0/1 Erkennung
\end{itemize}
\subsection{Schnittstellenereignistypen}
\subsubsection{Primitive des Dienstes}
\begin{itemize}
	\item Dienstname
	\item Dienstpritivname
	\item Grundform der Dienstleistung
\end{itemize}
\subsubsection{Dienstbeschreibung durch Zustandsdiagramme (18ff)}
\subsubsection{Protokollbeschreibung (23ff)}
\subsubsection{Point-to-Point Protocol PPP}
\begin{itemize}
\item Protokoll der Sicherungsschicht für Interet-Zugang, z.B. über Modem, ISDN, xDSL, ...
\item Protokoll der sicherungsschicht für Packet over SONET und ADSL
\item Network Control Protocol NCP
\item Link Control Protocol LCP
\item Funktionsweise basierend auf HDLC
\item Ablauf (29)
\end{itemize}
\subsubsection{ADSL}
\begin{itemize}
\item Überblick und Protokollstapel: verbindet Privathaushalte über bestehende Telefonleitung mit dem Internet
\item ATM (Asynchronous Transfer Mode)
\item AAL5
\item AAL Rahmen (31)
\end{itemize}