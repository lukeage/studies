% !TEX root = ../../Head.tex


\chapter{Vorlesung}

\section{Einführung}
\begin{itemize}
	\item Anwendungsfelder Rechnernetze (1.4)
	\begin{itemize}
		\item Geschäftsanwendungen - gemeinsame Nutzung von 							Resourcen
		\item Privatbereich - Informationszugriff (z.B. WWW, IM)
		\item Mobile Benutzer - Textnachrichten, ...
		\item Gesellschaftliche Aspekte - Copyright, Profile, ...
	\end{itemize}
	\item Client Server Modell (1.5)
	\item Peer-to-Peer Communication (1.6)
	\item Basis-Netzstruktur (1.7)
	\begin{itemize}
		\item Übertragungsmodi
		\begin{itemize}	
			\item Verbindungsorientiert
			\item Verbindungslos (z.B. IP)
			\item Leitungsvermittelt
			\item Paketvermittelt (flexibler, 												ressourcenschonend)
		\end{itemize}
	\end{itemize}
	\item Schichtenarchitektur - ISO/OSI Referenzmodell (1.8)
	\begin{itemize}
		\item International Organization for Standardization
		\item Open Systems Interconnection
		\item Schichtenübersicht auf 1.8 ff.
	\end{itemize}
	\item Integriertes Referenzmodell (Tanenbaum) (1.11)
	\begin{itemize}
		\item Protokollimplementierung oft abweichend vom Referenzmodell
	\end{itemize}
	\item Besipiel Datenübertragung (1.12)
	\item Schichteneffizienz (1.13)
	\item Dienste - Begriffsklärung (1.14)
	\begin{itemize}
		\item Beispiel Ablaufdiagramm (1.15)
	\end{itemize}
	\item Netzkopplung - Basis-Topologien
	\begin{itemize}
		\item Punkt-zu-Punkt-Kanäle (Unicast)
		\item Rundsendekanäle (Broadcast)
		\item Klassifizierung nach Ausdehnung (1.17)
		\begin{itemize}
			\item Pan - Personal Area Network
			\item LAN - Local Area Network
			\item MAN - Metropolitan Aria Network
			\item WAN - Wide Area Network (1.18)
		\end{itemize}
		\item Mobilität || Leistung (1.19)
		\item Konzepte - Layer-N-Gateway(1.20)
		\item Beispiel (1.21)
	\end{itemize}
	\item Internet(1.22 ff)
	\begin{itemize}
		\item Internet
		\begin{itemize}
			\item Geschichte des Internet (1.24 ff)
			\item Normen (1.26)
		\end{itemize}
		\item Intranet (1.22)
	\end{itemize}
\end{itemize}

\section{Bitübertragungsschicht}

\subsection{Nachrichtentechnische Kanäle}
\begin{itemize}
	\item Aufgabe: Physikalische Bitübertragung mittels Transformation in elektromagnetisches Signal
	\item Daten \(\to\) Kanal \(\rightsquigarrow\) Störeinflüsse \(\to\) Daten

\subsubsection{Kenngrößen (2.4 ff)}
	\begin{itemize}
		\item Bandbreite B: Breite des Frequenzbereichs eines Kanals, in dem ohne größere Dämpfung übertragen wird
		\item Baudrate %\( \text{BR} = \fraq{\text{Signalschritte}}{\textsc{s}} \)
		\item Bitrate
		\item Nyquist Theorrem \( b < 2 \cdot B \cdot ld(S) \)
		\begin{itemize}
			\item Erweiterung durch Shannon \( b < B \cdot ld(1+SNR)\)
			\item Kombination \(b < \text{min}(2 \cdot B \cdot ld(S)\text{ ; } B \cdot ld(1+SNR) )\)
		\end{itemize}		
	\end{itemize}

\subsubsection{Leitungscodes}
	\begin{itemize}
		\item Wie soll Folge von 0en und 1en übertragen werden?
		\item NRZ: \textbf{N}on-\textbf{R}eturn-to-\textbf{Z}ero  (2.6)
		\item Manchester-Codierung
		\item NRZI: NRZ-\textbf{I}nverted (2.7)
		\begin{itemize}
			\item Signaländerung bei 1, keine Signaländerung bei 0
			\item Vortei: hohe Netto-Datenrate
			\item Nachteil: Probleme bei langer Folge von Nullen
			\item Lösung: 4B/5B Code
			\begin{itemize}
				\item jeweils 4 Bits Daten werden auf 5-Bit-Muster abgebildet \(\to\) 25% Overhead (statt 100% wie bei Manchester
				\item durch 4B/5B-Code treten niemals mehr als 3 Nullen nacheinander auf
			\end{itemize}
		\end{itemize}
	\end{itemize}
\end{itemize}

\subsection{Übertragungsmedien}
\subsubsection{Elektrische Leitungen}
\begin{itemize}
	\item Twisted Pair (2.8)
	\begin{itemize}
		\item isolierte Kupferdräthe von 0,4 bis 1mm Stärke
		\item Paarweise verdrillt \(\to\) Reduzierung von Störungen
		\item Üblicherweise 4 Paare pro Kabel
		\item Mehrere Kilometer Reichweite, mehrere MBit/s, preiswert
		\item Signal aus Spannungsdifferenz zwischen den 2 Kabeln übertragen
		\item Cat 3
		\item Cat 5
		\item Cat 6
		\item Cat 7
	\end{itemize}
	\item Koaxialkabel (2.9)
	\begin{itemize}
		\item mehrere km, mehrere MBit/s, T-stecker ode rTap
		\item 50-Ohm-Kabel: für digitale Übertragung
		\item 75-Ohm-Kabel: für analoge Übertragung und Kabelfernsehen
		\item Kabelfernsehen \(\to\) Breitband-Koaxialkabel, häufig mit analoger Übertragung bis ca. 1 GHz, bidirektionaler Ausbau für Internet-Zugang via Kabel
	\end{itemize}
\end{itemize}

\subsubsection{Optische Leitungen und Sichtverbindung}
\begin{itemize}
	\item Optische Leitungen
	\begin{itemize}
		\item Lichtwellenleiter (LWL) / "Glasfaser"
		\begin{itemize}
			\item bis TBit/s-Bereich, über viele km Entfernung
			\item Monomodefaser: nur eine ausbreitungsfähige Wellenform
			\item Multimodefaser: verschiedene ausbreitungsfähige Wellenformen
			\item Gradientenfaser: schrittweise Änderung des Brechungsindex
		\end{itemize}
	\end{itemize}
\end{itemize}

\subsubsection{Sichtverbindung}
\begin{itemize}
	\item Infrarotverbindung
	\item Richtfunkstrecken
\end{itemize}

\subsubsection{Satelliten / Zellularfunk (2.11)}
\begin{itemize}
	\item Satelliten
	\begin{itemize}
		\item Getrennte Aufwärts-/Abwärtsbänder
		\item Bandbreite von 500MHz, z.B. in mehrere 50 MBit/s - Kanäle oder 800 digitale Sprachkanäle mit 64 kBit/s
		\item Zuordnung kurzer Zeitabschnitte zu einzelnen Kanälen (Zeitmultiplex)
		\item Lange Laufzeiten (ca. 250 bis 300ms)
	\end{itemize}
	\item Zellularfunk
	\begin{itemize}
		\item Aufteilung eines geographischen Bereichs in Funkzellen mit spezifischen Frequenzbändern
		\item Beispiel: GSM (Global System for Mobile Communication)
	\end{itemize}
\end{itemize}

\subsubsection{Strukturierte Verkabelung (2.12}
\begin{itemize}
	\item Ziel: Systematische, gut wartbare und erweiterbare Kabelinfrastruktur
	\item Trennung in drei wesentliche Bereiche (jeweils sternförmig hierarchisch)
	\begin{itemize}
		\item Primärebene
		\item Sekundärebene
		\item Tertiärebene
	\end{itemize}
\end{itemize}

\subsection{Mehrfachnutzung von Kanälen}
\subsubsection{Frequenzmultiplex (2.13)}
\begin{itemize}
	\item getrennte Frequenzbänder (mit z.B. 3000 Hz) und zwischengeschaltete Sperrbänder (mit z.B. 500 Hz)
\end{itemize}

\subsubsection{Orthogonales Frequenzmultiplex (Orthogonal FDM, OFDM)}
\begin{itemize}
	\item Überlaguerung der Kanäle ohne Sperrbänder \(\to\) effizienter
	\item Empfänger: Trennung der Signale mehrerer Bänder durch schnelle Fouriertransformation
	\item Einsatz: Wlan, Kabelnetze, 4G Mobilfunk, LTE, ...
\end{itemize}

\subsubsection{Zeitmultiplex (2.14)}
\begin{itemize}
	\item Zyklische Kanalzuteilung
\end{itemize}

\subsubsection{Statistisches Zeitmultiplex}
\begin{itemize}
	\item flexible Zuteilung nach Bedarf
\end{itemize}

\subsubsection{Codemultiplex (CDM, 2.15)}
\begin{itemize}
	\item Didizierte (Kodierungs-)Codes pro Teilnehmerpaar
\end{itemize}

\subsubsection{Wellenlängenmultiplex (WDM)}
\begin{itemize}
	\item Variation von Frequenzmultiplex, indem direkte optische Einkopplung mehrerer \\Lichtwellenleiter (mit Licht unterschiedlicher Wellenlängen) in einen besonders leistungsfähigen Lichtwellenleiter erfolgt
	\item entsprechende Wiederauskopplung im Zielsystem
\end{itemize}
\subsection{Datenübertragung}
\subsubsection{Signalklassen (2.16)}
\begin{itemize}
	\item Wert/Zeit kontinuierlich \(\leftrightarrow \) Wert/Zeit diskret
	\item Beispiele (2.17)
	\begin{itemize}
		\item Wert- und zeitkontinuierlich: analoges Telefon
		\item Wertkontinuierlich, zeitdiskret: Prozesssteuerung mit periodischen Messpunkten
		\item Wertdiskret, zeitkontinuerlich: digitale Temperaturanzeige
		\item Wert- und zeitdiskret: digitale Übertragung mit isochronem Taktmuster; z.B. Sprachübertragung über digitale Kanäle
	\end{itemize}
\end{itemize}
\subsubsection{Beispiel: Telefonsystem (2.18)}
\subsubsection{Sprachübertragung über digitale Kanäle (2.19)}
\begin{itemize}
	\item Analoge Eingangssignale (Sprache) vor Übertragung im Kernnetz zu digitalisieren: Codec (Coder-Decoder)
	\item Basis: Abtasttheorem nach Shannon \( f_(A) > 2\cdot f_(G) \)
	\item PCM: Pulse Code Modulation
	\begin{itemize}
		\item Bsp.: Grenzfrequenz (Telefon) : 3400 z; Abtastfrequenz: 8000 Hz
		\item logarithmische Quantisierungsintervalle \(\to\) Quantisierungsfehler begrenzen
	\end{itemize}
\end{itemize}
\subsubsection{Datenübertragung über analoge Kanäle}
\begin{itemize}
	\item Modem: Übertragung digitaler Signale über analoge (2.20) Telefonverbindung
	\begin{itemize}
		\item Problem: Nicht direkt möglich wegen kapazitiver und induktiver Einflüsse
	\end{itemize}
	\item Amplitudenabtastung
	\item Periodenabtastung
	\item Phasenabtastung
	\begin{itemize}
		\item Ziel: Deutlich höhere Übertragungsleistung durch gleichzeitige Anwendung mehrerer Modulationsverfahren (2.21)
		\item Beispiele
		\begin{itemize}
			\item QPSK
			\item QAM 16
			\item QAM 64
		\end{itemize}
	\end{itemize}
\end{itemize}

\subsection{Beispieltechnologien}
\subsubsection{Digital Subscriber Line (DSL, 2.22)}
\begin{itemize}
	\item digitaler Netzzugang über herkömmliche Telefonleitungen
	\item Datenübertragung und Telefondienst gleichzeitig nutzbar
	\item Realisierung durch Nutzung höherer Frequenzbereiche
	\item hohe Datenraten, meist asymmetrisch (ADSL) bzgl. Up-/Downlink
	\item weitere Varianten:
	\begin{itemize}
		\item VDSL (Very High Bitrate) : nur über kurze Entfernungen
		\item SDSL (Symmetric): GLEICHE dATENRATE AUF Up-/Downlink
	\end{itemize}
	\item Signaltrennung (Telefon/Daten) und Modulation (basierend auf QAM, 2.23)
	\begin{itemize}
		\item CAS (Carrierless Amplitude / Pase System)
		\item DMT (Discrete Multitone)
	\end{itemize}
\end{itemize}
\subsection{Digitaler Netzzugang über Kabelmodem}
\begin{itemize}
	\item Signaltrennung zwischen Kabelfernsehen und Daten:
	\begin{itemize}
		\item Umwidmung einzelner TV-Kanäle in Datenkanäle
		\item Rückkanalfähige Verstärker erforderlich
		\item Datenraten theoretisch bis ca. 36 MBit/s, aber "Shared Medium", d.h. abhängig von der Zahl der Teilnehmer geringere Datenrate
	\end{itemize}
\end{itemize}
