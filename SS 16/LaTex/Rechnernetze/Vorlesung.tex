% !TEX root = ../../Head.tex


\chapter{Vorlesung}

\section{Einführung}
\begin{itemize}
	\item Anwendungsfelder Rechnernetze (1.4)
	\begin{itemize}
		\item Geschäftsanwendungen - gemeinsame Nutzung von 							Resourcen
		\item Privatbereich - Informationszugriff (z.B. WWW, IM)
		\item Mobile Benutzer - Textnachrichten, ...
		\item Gesellschaftliche Aspekte - Copyright, Profile, ...
	\end{itemize}
	\item Client Server Modell (1.5)
	\item Peer-to-Peer Communication (1.6)
	\item Basis-Netzstruktur (1.7)
	\begin{itemize}
		\item Übertragungsmodi
		\begin{itemize}	
			\item Verbindungsorientiert
			\item Verbindungslos (z.B. IP)
			\item Leitungsvermittelt
			\item Paketvermittelt (flexibler, 												ressourcenschonend)
		\end{itemize}
	\end{itemize}
	\item Schichtenarchitektur - ISO/OSI Referenzmodell (1.8)
	\begin{itemize}
		\item International Organization for Standardization
		\item Open Systems Interconnection
		\item Schichtenübersicht auf 1.8 ff.
	\end{itemize}
	\item Integriertes Referenzmodell (Tanenbaum) (1.11)
	\begin{itemize}
		\item Protokollimplementierung oft abweichend vom Referenzmodell
	\end{itemize}
	\item Besipiel Datenübertragung (1.12)
	\item Schichteneffizienz (1.13)
	\item Dienste - Begriffsklärung (1.14)
	\begin{itemize}
		\item Beispiel Ablaufdiagramm (1.15)
	\end{itemize}
	\item Netzkopplung - Basis-Topologien
	\begin{itemize}
		\item Punkt-zu-Punkt-Kanäle (Unicast)
		\item Rundsendekanäle (Broadcast)
		\item Klassifizierung nach Ausdehnung (1.17)
		\begin{itemize}
			\item Pan - Personal Area Network
			\item LAN - Local Area Network
			\item MAN - Metropolitan Aria Network
			\item WAN - Wide Area Network (1.18)
		\end{itemize}
		\item Mobilität || Leistung (1.19)
		\item Konzepte - Layer-N-Gateway(1.20)
		\item Beispiel (1.21)
	\end{itemize}
	\item Internet(1.22 ff)
	\begin{itemize}
		\item Internet
		\begin{itemize}
			\item Geschichte des Internet (1.24 ff)
			\item Normen (1.26)
		\end{itemize}
		\item Intranet (1.22)
	\end{itemize}
\end{itemize}

\section{Bitübertragungsschicht}

\subsection{Nachrichtentechnische Kanäle}
\begin{itemize}
	\item Aufgabe: Physikalische Bitübertragung mittels Transformation in elektromagnetisches Signal
	\item Daten \(\to\) Kanal \(\rightsquigarrow\) Störeinflüsse \(\to\) Daten

\subsubsection{Kenngrößen (2.4 ff)}
	\begin{itemize}
		\item Bandbreite B: Breite des Frequenzbereichs eines Kanals, in dem ohne größere Dämpfung übertragen wird
		\item Baudrate %\( \text{BR} = \fraq{\text{Signalschritte}}{\textsc{s}} \)
		\item Bitrate
		\item Nyquist Theorrem \( b < 2 \cdot B \cdot ld(S) \)
		\begin{itemize}
			\item Erweiterung durch Shannon \( b < B \cdot ld(1+SNR)\)
			\item Kombination \(b < \text{min}(2 \cdot B \cdot ld(S)\text{ ; } B \cdot ld(1+SNR) )\)
		\end{itemize}		
	\end{itemize}

\subsubsection{Leitungscodes}
	\begin{itemize}
		\item Wie soll Folge von 0en und 1en übertragen werden?
		\item NRZ: \textbf{N}on-\textbf{R}eturn-to-\textbf{Z}ero  (2.6)
		\item Manchester-Codierung
		\item NRZI: NRZ-\textbf{I}nverted (2.7)
		\begin{itemize}
			\item Signaländerung bei 1, keine Signaländerung bei 0
			\item Vortei: hohe Netto-Datenrate
			\item Nachteil: Probleme bei langer Folge von Nullen
			\item Lösung: 4B/5B Code
			\begin{itemize}
				\item jeweils 4 Bits Daten werden auf 5-Bit-Muster abgebildet \(\to\) 25% Overhead (statt 100% wie bei Manchester
				\item durch 4B/5B-Code treten niemals mehr als 3 Nullen nacheinander auf
			\end{itemize}
		\end{itemize}
	\end{itemize}
\end{itemize}

\subsection{Übertragungsmedien}
\subsubsection{Elektrische Leitungen}
\begin{itemize}
	\item Twisted Pair (2.8)
	\begin{itemize}
		\item isolierte Kupferdräthe von 0,4 bis 1mm Stärke
		\item Paarweise verdrillt \(\to\) Reduzierung von Störungen
		\item Üblicherweise 4 Paare pro Kabel
		\item Mehrere Kilometer Reichweite, mehrere MBit/s, preiswert
		\item Signal aus Spannungsdifferenz zwischen den 2 Kabeln übertragen
		\item Cat 3
		\item Cat 5
		\item Cat 6
		\item Cat 7
	\end{itemize}
	\item Koaxialkabel (2.9)
	\begin{itemize}
		\item mehrere km, mehrere MBit/s, T-stecker ode rTap
		\item 50-Ohm-Kabel: für digitale Übertragung
		\item 75-Ohm-Kabel: für analoge Übertragung und Kabelfernsehen
		\item Kabelfernsehen \(\to\) Breitband-Koaxialkabel, häufig mit analoger Übertragung bis ca. 1 GHz, bidirektionaler Ausbau für Internet-Zugang via Kabel
	\end{itemize}
\end{itemize}

\subsubsection{Optische Leitungen und Sichtverbindung}
\begin{itemize}
	\item Optische Leitungen
	\begin{itemize}
		\item Lichtwellenleiter (LWL) / "Glasfaser"
		\begin{itemize}
			\item bis TBit/s-Bereich, über viele km Entfernung
			\item Monomodefaser: nur eine ausbreitungsfähige Wellenform
			\item Multimodefaser: verschiedene ausbreitungsfähige Wellenformen
			\item Gradientenfaser: schrittweise Änderung des Brechungsindex
		\end{itemize}
	\end{itemize}
\end{itemize}

\subsubsection{Sichtverbindung}
\begin{itemize}
	\item Infrarotverbindung
	\item Richtfunkstrecken
\end{itemize}

\subsubsection{Satelliten / Zellularfunk (2.11)}
\begin{itemize}
	\item Satelliten
	\begin{itemize}
		\item Getrennte Aufwärts-/Abwärtsbänder
		\item Bandbreite von 500MHz, z.B. in mehrere 50 MBit/s - Kanäle oder 800 digitale Sprachkanäle mit 64 kBit/s
		\item Zuordnung kurzer Zeitabschnitte zu einzelnen Kanälen (Zeitmultiplex)
		\item Lange Laufzeiten (ca. 250 bis 300ms)
	\end{itemize}
	\item Zellularfunk
	\begin{itemize}
		\item Aufteilung eines geographischen Bereichs in Funkzellen mit spezifischen Frequenzbändern
		\item Beispiel: GSM (Global System for Mobile Communication)
	\end{itemize}
\end{itemize}

\subsubsection{Strukturierte Verkabelung (2.12}
\begin{itemize}
	\item Ziel: Systematische, gut wartbare und erweiterbare Kabelinfrastruktur
	\item Trennung in drei wesentliche Bereiche (jeweils sternförmig hierarchisch)
	\begin{itemize}
		\item Primärebene
		\item Sekundärebene
		\item Tertiärebene
	\end{itemize}
\end{itemize}

\subsection{Mehrfachnutzung von Kanälen}
\subsubsection{Frequenzmultiplex (2.13)}
\begin{itemize}
	\item getrennte Frequenzbänder (mit z.B. 3000 Hz) und zwischengeschaltete Sperrbänder (mit z.B. 500 Hz)
\end{itemize}

\subsubsection{Orthogonales Frequenzmultiplex (Orthogonal FDM, OFDM)}
\begin{itemize}
	\item Überlaguerung der Kanäle ohne Sperrbänder \(\to\) effizienter
	\item Empfänger: Trennung der Signale mehrerer Bänder durch schnelle Fouriertransformation
	\item Einsatz: Wlan, Kabelnetze, 4G Mobilfunk, LTE, ...
\end{itemize}

\subsubsection{Zeitmultiplex (2.14)}
\begin{itemize}
	\item Zyklische Kanalzuteilung
\end{itemize}

\subsubsection{Statistisches Zeitmultiplex}
\begin{itemize}
	\item flexible Zuteilung nach Bedarf
\end{itemize}

\subsubsection{Codemultiplex (CDM, 2.15)}
\begin{itemize}
	\item Didizierte (Kodierungs-)Codes pro Teilnehmerpaar
\end{itemize}

\subsubsection{Wellenlängenmultiplex (WDM)}
\begin{itemize}
	\item Variation von Frequenzmultiplex, indem direkte optische Einkopplung mehrerer \\Lichtwellenleiter (mit Licht unterschiedlicher Wellenlängen) in einen besonders leistungsfähigen Lichtwellenleiter erfolgt
	\item entsprechende Wiederauskopplung im Zielsystem
\end{itemize}
\subsection{Datenübertragung}
\subsubsection{Signalklassen (2.16)}
\begin{itemize}
	\item Wert/Zeit kontinuierlich \(\leftrightarrow \) Wert/Zeit diskret
	\item Beispiele (2.17)
	\begin{itemize}
		\item Wert- und zeitkontinuierlich: analoges Telefon
		\item Wertkontinuierlich, zeitdiskret: Prozesssteuerung mit periodischen Messpunkten
		\item Wertdiskret, zeitkontinuerlich: digitale Temperaturanzeige
		\item Wert- und zeitdiskret: digitale Übertragung mit isochronem Taktmuster; z.B. Sprachübertragung über digitale Kanäle
	\end{itemize}
\end{itemize}
\subsubsection{Beispiel: Telefonsystem (2.18)}
\subsubsection{Sprachübertragung über digitale Kanäle (2.19)}
\begin{itemize}
	\item Analoge Eingangssignale (Sprache) vor Übertragung im Kernnetz zu digitalisieren: Codec (Coder-Decoder)
	\item Basis: Abtasttheorem nach Shannon \( f_(A) > 2\cdot f_(G) \)
	\item PCM: Pulse Code Modulation
	\begin{itemize}
		\item Bsp.: Grenzfrequenz (Telefon) : 3400 z; Abtastfrequenz: 8000 Hz
		\item logarithmische Quantisierungsintervalle \(\to\) Quantisierungsfehler begrenzen
	\end{itemize}
\end{itemize}
\subsubsection{Datenübertragung über analoge Kanäle}
\begin{itemize}
	\item Modem: Übertragung digitaler Signale über analoge (2.20) Telefonverbindung
	\begin{itemize}
		\item Problem: Nicht direkt möglich wegen kapazitiver und induktiver Einflüsse
	\end{itemize}
	\item Amplitudenabtastung
	\item Periodenabtastung
	\item Phasenabtastung
	\begin{itemize}
		\item Ziel: Deutlich höhere Übertragungsleistung durch gleichzeitige Anwendung mehrerer Modulationsverfahren (2.21)
		\item Beispiele
		\begin{itemize}
			\item QPSK
			\item QAM 16
			\item QAM 64
		\end{itemize}
	\end{itemize}
\end{itemize}

\subsection{Beispieltechnologien}
\subsubsection{Digital Subscriber Line (DSL, 2.22)}
\begin{itemize}
	\item digitaler Netzzugang über herkömmliche Telefonleitungen
	\item Datenübertragung und Telefondienst gleichzeitig nutzbar
	\item Realisierung durch Nutzung höherer Frequenzbereiche
	\item hohe Datenraten, meist asymmetrisch (ADSL) bzgl. Up-/Downlink
	\item weitere Varianten:
	\begin{itemize}
		\item VDSL (Very High Bitrate) : nur über kurze Entfernungen
		\item SDSL (Symmetric): GLEICHE dATENRATE AUF Up-/Downlink
	\end{itemize}
	\item Signaltrennung (Telefon/Daten) und Modulation (basierend auf QAM, 2.23)
	\begin{itemize}
		\item CAS (Carrierless Amplitude / Pase System)
		\item DMT (Discrete Multitone)
	\end{itemize}
\end{itemize}
\subsection{Digitaler Netzzugang über Kabelmodem}
\begin{itemize}
	\item Signaltrennung zwischen Kabelfernsehen und Daten:
	\begin{itemize}
		\item Umwidmung einzelner TV-Kanäle in Datenkanäle
		\item Rückkanalfähige Verstärker erforderlich
		\item Datenraten theoretisch bis ca. 36 MBit/s, aber "Shared Medium", d.h. abhängig von der Zahl der Teilnehmer geringere Datenrate
	\end{itemize}
\end{itemize}

\section{Netztechnologie 1}
\subsection{Medienzugriff}
\subsubsection{ALOHA Protokoll}
\begin{itemize}
	\item historisches Paketfunknetz, University of Hawaii, seit 1979
	\item dezentrale Stationen, Kommunikation über Zentrale
	\item unkoordiniertes Wettbewerbsverfahren (stochastisch)
	\item Kollision auf \(f_{1}\) bei Zentrale, da Senden stets möglich
	\item Fehlerbehandlung durch Wiederholung, falls nach Zeit \(t\) keine Quittung auf \(f_{2}\)
	\item kein Mithören während des Sendevorgangs
\end{itemize}
\subsubsection{ALOHA Beispiel}
\begin{itemize}
	\item Pure ALOHA: Max. etwa 18 Prozent des Kanaldurchsatzes
	\item Slotted ALOHA: Max. etwa 36 Prozent des Kanaldurchsatzes (3.6)
\end{itemize}
\subsubsection{CSMA-Verfahren}
\begin{itemize}
	\item kein Funk, sondern Coaxialkabel
	\item Abhören vor Senden (CSMA - Carrier Sense Multiple Acces)
	\item Trotzdem Kollision möglich: (1-persistent CSMA, immer sendebereit)
	\item nonpersistent CSMA: belegter Kanal wird nicht sofort erneut abgehört, erst nach zufällig verteiltem Zeitintervall; dadurch geringere Kollisionswahrscheinlichkeit
	\item p-persistent CSMA (slotted): Prüfe Kanal, sende mit Wahrscheinlichkeit p, warte sonst 1 Slot und prüfe wieder
\end{itemize}
\subsubsection{Bewertung der Verfahren (3.8)}
\subsubsection{CSMA mit "Collision Detection" (CD)}
\begin{itemize}
	\item Mithören während des Sendevorgangs
	\item Kollisionserkennung dadurch schneller möglich (ohne Warten auf Quittung)
	\item Funktioniert für ein gemeinsam genutztes Kommunikationsmedium ... (z.B. gemeinsames Kabel bei IEEE 802.3, Lukf bei IEEE 802.11, etc.)
	\item ... mit mindestens einer Station (Kollisionen mit sich selbst können erkannt werden, z.B. durch Signalreflexion am offenen Kabelende
\end{itemize}
\subsubsection{CSMA/CD-Verfahren Beispiel (3.10 ff.)}
\subsection{Ethernet}
\subsubsection{IEEE 802.3}
\begin{itemize}
	\item Zugriffsverfahren: 1-persistent CSMA/CD, in Hardware auf Ethernet-Karte realisiert
	\item Datenrate der Basistechnologie: 10 MBit/s
	\item Segmentlänge: 500m
	\item Kabel der Kategorie 5 oder höher bzw. Lichtwellenleiter (dann auch deutlich größere räumliche Ausdehnungen möglich)
	\item heute grundsätzlich mit Switches und Duplex-Betrieb im Einsatz
	\item dennoch Kollisionsbehandlung generell mit eingebaut:
	\begin{itemize}
		\item warte s Slots nach Kollision, s zwischen 0 und \(2^n-1\) bei n vorherigen Kollisionen zufällig gewählt
	\end{itemize}
\end{itemize}
\subsubsection{Rahmenstruktur (3.14)}
\begin{itemize}
	\item Präambel erlaubt Synchronisation mit Empfänger
	\item EtherType/Size
	\begin{itemize}
		\item \(\le 1500 \to \) Länge des Datenfelds
		\item \(\ge 1536 \to \) Typ der Daten (z.B. IP, IPv6, etc.), Länge der Daten nicht spezifiziert \(\to\) Interframe Gap als Begrenzer
	\end{itemize}
	\item Pad zum Auffüllen auf minimale Rahmenlänge wegen Kollisionsverfahren
	\item Prüfsumme: CRC (Cyclic Redundancy Check), ohne Präambel und SFD
\end{itemize}
\subsubsection{Fast- / Gigabit Ethernet}
\begin{itemize}
	\item Fast Ethernet
	\begin{itemize}
		\item 1995 als IEEE 802.3u standardisiert
		\item Datenrate 100 MBit/s
		\item Segmentlänge: 100m bei Kupferkabel, 2km bei Lichtwellenleiter
		\item Kompatibilität zu Ethernet und Cat-3-Kabel, noch CSMA/CD unterstützt aber keine Multidrop-Kabel mehr
	\end{itemize}
	\item Gigabit Ethernet
	\begin{itemize}
		\item 1999 als IEEE 802.3ab standardisiert
		\item Datenrate 1 GBit/s
		\item Vollduplex (Standard): kein CSMA/CD mehr \(\to\) keine Beschränkung der Kabellänge
		\item Halbduplex: Layer-1-Kopplung über Hub; CSMA/CD mit Modifikationen: 
		\begin{itemize}
			\item Padding - Rahmen immer auf 512 Byte auffüllen
			\item Frame-Bursting - mehrere Rahmen in einem Ethernet-Frame übertragen
		\end{itemize}
	\end{itemize}
	\item 10 / 100 GBit/s Ethernet
	\begin{itemize}
		\item für optische Verbindungen in WANs \(\to\) siehe Kapitel 4
	\end{itemize}
\end{itemize}
\subsubsection{Ethernet-Varianten für LAN (3.16)}
\subsubsection{Switched Ethernet: Beispiel (3.17)}
\begin{itemize}
	\item parallele Vermittlung aller Verkehrsströme durch Switch-Hardware
	\item Vorteil: Keine Kollisionen, jeder Station steht die volle Ethernet-Datenrate zur Verfügung \(\Rightarrow\) Ethernet wird vom "Shared Medium" zum "Switched MEdium"
	\item Aufteilung der Stationen and einem oder mehreren Switches in unterschiedliche virtuelle lokale Netze (VLAN) möglich \(\Rightarrow\) Sicherheitszonen
\end{itemize}

\subsection{Switches in der Sicherungsschicht}
Ziele (3.18):
\begin{itemize}
	\item parallele Vermittlung durch Switches, sequentiell duch Bridges (veraltet)
	\item Trennung organisatorischer Bereiche/verschieden Verkehrsströme
	\item Zuverlässigkeit und Sicherheit (gegen Störsignale und unberechtigte Weiterleitung)
	\item Begrenzung der Netzlast durch selektives Weiterleiten von Nachrichten
\end{itemize}
\subsubsection{Modell (3.19)}
\subsubsection{Transparent Bridges / Switches (3.20)}
\begin{itemize}
	\item Selbstlernend: Automatischer Aufbau von Routing-Tabellen
	\item Topologie-Erkennung durch Quelladressen, schrittweiser Tabellenaufbau
	\item Fluten, falls Zielrechner noch unbekannt
	\item Löschen von Einträgen nach bestimmter Zeit zur Anpassung an Topologieänderungen
\end{itemize}
\subsubsection{Spanning Tree (3.21)}
\begin{itemize}
	\item Problem: Mehrfachwege \(\to\) Endlosschleifen
	\item Lösung: Aufbau eines "überspannenden Baumes" mit eindeutigen Wegen durch dezentralen Algorithmus / kürzester Weg zur Wurzel
\end{itemize}
\subsubsection{Interne Realisierung (3.22)}
\begin{itemize}
	\item Parallele Vermittlung mehrerer Eingangs- an mehrere Ausgangsports
	\item Hohe Leistung, unterstützt durch Hardware-Realisierung
	\item Store-and-Forward-Switch: Gesamtes Frame wird im Switch zwischengespeichert, die Prüfsumme wird kontrolliert und erst dann wird weitergeleitet \(\Rightarrow\) einfach; Pufferung und Datenratenanpassung
	\item Cut-Through-Switch: Andommende Frames werden nach Prüfung der Zieladresse sofort weitergeleitet \(\Rightarrow\) effizienter, kürzere Verzögerung, aber problematisch bei unterschiedlichen Datenraten und bei Fehlern
\end{itemize}
\subsubsection{VLAN - Virtual Local Area Network (2.23)}
\begin{itemize}
	\item Motivation:
	\begin{itemize}
		\item Flexibilität: Änderung der Zuordnung von Geräten zu lokalen Netzen ohne neue Verkabelung
		\item Sicherheits- und Performance-Aspekte
	\end{itemize}
\end{itemize}
\subsubsection{Port-basiertes VLAN (3.24)}
\begin{itemize}
	\item Jeder Port eines Switches wird einem VLAN zugeordnet
	\item Ports können nur Mitglied eines VLANs sein
	\item redundante Links zwischen Switches benötigt
\end{itemize}
\subsubsection{Tag-basiertes VLAN - IEEE 802.1Q (3.25)}
\begin{itemize}
	\item Transport mehrerer VLAN-Pakete über einen Link \(\to\) Tagging der Pakete
	\item IEEE 802.1Q - Ergänzung des Ethernet-Headers - VLAN-Tag
	\item letzter VLAN-Fähiger Switch entfernt das VLAN-Tag wieder \(\to\) Kompatibilität
	\item VLAN Identifier = 12 Bit
	\item andere Felder (Priority und CFI) nicht für VLAN genutzt
\end{itemize}
\subsection{Drahtlose Netze für PAN und LAN (3.26)}
\subsubsection{WLAN: IEEE 802.11 (3.27)}
\subsubsection{802.11 - Medienzugriff mit CSMA/CA (3.28) ff.}
\begin{itemize}
	\item RTS/CTS - Request to Send / Clear to Send
	\item Hidden terminal: A kann C wegen begrenzter Funkreichweite nicht hören
	\begin{itemize}
		\item A sendet RTS-Signal an B, und B sendet dann CTS
		\item Alle anderen möglichen Sender (C) erhalten das CTS-Signal und stellen ihren Sendevorgang zurück
	\end{itemize}
	\item Exposed terminal (unnötiges Warten, hier durch B bei Senden nach links)
	\begin{itemize}
		\item C sendet RTS an möglichen anderen Empfänger
		\item Falls dieser beriet, erhält C das CTS und kann übertragen (unabhängig von B)
	\end{itemize}
\end{itemize}
\subsubsection{Bluetooth}
\begin{itemize}
		\item drahtlose Ad-Hoc-Piconetze (<10m), billige Ein-Chip-Lösung
		\item offener Standard: IEEE 802.15.1
		\item Einsatzgebiete:
		\begin{itemize}
			\item Verbindung von Perpheriegeräten
			\item Unterstützung von Ad-Hoc-Netzen
			\item Verbindung verschiedener Netze (z.B. drahtloses Headset mit GSM)
		\end{itemize}
		\item Frequenzband im 2,4 GHz- Bereich; Integrierte Sicherheitsverfahren \\(128-Bit-Verschlüsselung)
		\item Datenraten:
		\begin{itemize}
			\item 433,9 kBit/s asynchronous-symmetrical
			\item 723,2 kBit/s / 57,6 KBit/s asynchronous-asymmetrical
			\item 64 kBit/s synchronous, voice service
			\item Erweiterungen bis zu 20 Mbit/s (IEEE 802.15.3a: UWB (Ultra Wide Band)
		\end{itemize}
\end{itemize}
\subsubsection{ZigBee (3.30}
\subsubsection{RFID - Radio Frequency Identification (3.31)}
\begin{itemize}
	\item Klasse-1-Tags:
	\begin{itemize}
		\item bestehen aus Antenne und RFID-Chip
		\item 96-Bit-Identifikator, kleiner Speicher, passiv
		\item geringer Preis, lässt sich z.B: auf Produkte aufkleben
	\end{itemize}
	\item Lesegerät:
	\begin{itemize}
		\item aktiv, leistungsfähig, MAC-Protokolle
		\item sendet Trägersignal, wird von Tag reflektiert
	\end{itemize}
	\item Backscatter: Tag überlagert das Trägersignal mit eigenen zu sendenden Bits \(\to\) Lesegerät filtert dies Bits aus
	\item Mehrfachzugriff: modifizierte Version von Slotted ALOHA
\end{itemize}
\subsubsection{NFC - Near Field Communication}
\begin{itemize}
	\item kontaktloser Datenaustausch über Kürzeste Distanzen (4cm)
	\begin{itemize}
		\item Auflegen/anlegen des Transmitters an Lesegerät erforderlich
	\end{itemize}
	\item Datenübertragunsrate bis zu 424 kBit/s
	\item Übertragung
	\begin{itemize}
		\item verbindungslos: passive RFID-Tags
		\item verbindungsorientiert: aktive Transmitter (z.B. Smartphone)
	\end{itemize}
	\item mögliche Anwendungen
	\begin{itemize}
		\item Bezahlung per Smartphone oder Smartcard
		\item Smartphone als Türschlüssel
	\end{itemize}
	\item Kritik
	\begin{itemize}
		\item Distanz als Sicherheitsfeature ungeeignet (durch große Antennen bis zu 1m möglich) 
		\item NFC-Sicherheitsmechanismen unzureichend
	\end{itemize}
\end{itemize}