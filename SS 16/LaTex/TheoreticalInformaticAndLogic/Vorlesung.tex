%!TEX root = ../../head.tex

\chapter{Vorlesung}
\section{Prädikatenlogik erster Stufe}

\begin{itemize}
\item Syntax
\begin{itemize}
	\item Ein Alphabet der Prädikatenlogik besteht aus ... (2)
	\item forall heist universeller Quantor, exists heißt							existenzieller Quantor
	\item Funktions- und Relationssymbolen ist eine Stelligkeit n el N			\item Nullstellige Funktionssymbole werden als ... (3)
\end{itemize}
\item Terme
\begin{itemize}
	\item Definition 4.2 prädikatenlogische Terme (4)
	\item Ein Term ist abgeschlossen oder grundinstanziiert, wenn in ihm 			keine Variablen vorkommen
	\item Die Menge der abgeschlossenen Terme wird mit \textit{T}						(\textit{F}) bezeichnet
\end{itemize}
\item Prädikatenlogische Atome (5)
\item Prädikatenlogische Formeln (6)
\begin{itemize}
	\item prädikatenlogische Formeln
\end{itemize}
\item Strukturelle Rekursion 
\begin{itemize}
	\item Rekursionssätze lassen sich für \textit{T}(\textit{F},					\textit{V}) und \textit{L}(\textit{R},\textit{F},\textit{V}) 				formulieren 
	\item Es gibt genau eine Funktion \textit{foo} die die folgenden Bedingungen 				erfüllt: (7)
	\begin{itemize}
		\item Rekursionsanfang
		\item Rekursionsschritt
	\end{itemize}
	\item Beispiele (8/9)
\end{itemize}
\end{itemize}

\section{Prädikatenlogik erster Stufe}
\begin{itemize}
	\item Strukturelle Induktion
	\begin{itemize}
		\item Induktionssätze lassen sich für T(F,V) und L(R,F,V) 						formulieren
		\item jeder Term besitzt die Eigenschaft E, wenn: (10)
		\item analog für prädikatenlogische Formeln
	\end{itemize}
	\item Aufgabe (11)
	\begin{itemize}
		\item Beweisen Sie, dass \begin{math}\forall F \in L(R,F,V) 			\end{math} die Aussage \begin{math} l'(m(F))\ge l(F) \end{math} gilt 
	\end{itemize}
	\item Teilterme und Teilformeln (12)
	\begin{itemize}
		\item Die Def. 3.8 lässt sich auf Terme und Formeln übertragen
		\item Beispiel
	\end{itemize}
	\item Freie und gebundene Vorkommen einer Variablen (13)
	\begin{itemize}
		\item Def. 4.5 Die \textbf{freien Vorkommen einer Variablen} in 				einer prädikatenlogischen Formel sind wie folgt definiert: 					(13)
	\end{itemize}
	\item Abgeschlossene Terme und Formeln (14)
	\begin{itemize}
		\item nach Def. 4.2: Ein abgeschlossener Term ist ein Term, in 						dem	keine Variable vorkommt
		\item Def. 4.6 Eine abgeschlossene Formel (oder kurz 							ein Satz) der Sprache  \textit{L(R,F,V)} ist eine Formel der 			Sprache \textit{L(R,F,V)}, in der jedes Vorkommen einer 					Variablen gebunden ist
	\end{itemize}
	\item Substitutionen (19)
	\begin{itemize}
		\item Def. 4.7: Eine \textbf{Substitution} ist eine Abbildung 					\begin{math} \sigma : V \to T(F,V)\end{math}, die bis auf 					endlich viele Stellen mit der Identitätsabbildung 							übereinstimmt
		\item Beispiel
	\end{itemize}
	\item Instanzen
	\begin{itemize}
		\item Statt \begin{math} \sigma(X) \end{math} schreiben wirn in 				der Folge \begin{math} X\sigma \end{math}
		\item Def. 4.8: Sei sigma eine Substitution \begin{math} \sigma 				: V \to T(F,V)\end{math} kann wie folgt zu einer Abbildung 					\begin{math} \sigma dach: T(F,V) \to T(F,V)\end{math} 						erweitert werden: (25)
		\item Grundinstanz
		\item Proposition
	\end{itemize}
	\item Komposition von Substitutionen
	\begin{itemize}
		\item Def. 4.10: Seien \(\sigma\) und \(\theta\) zwei 							Substitutionen Die 	Komposition \(\sigma\theta\) von 						\(\sigma\) und \(\theta\) ist die Substitution: (30)
		\item Aufgaben
	\end{itemize}
\end{itemize}

\section{Syntax/Substitutionen}
\subsection{Komposition von Substitutionen}
\subsubsection{Korollar 4.11}
Für jede Substitution \(\sigma\) gilt \(\epsilon\sigma = \sigma = \sigma\epsilon \)
\subsubsection{Proposition 4.12}
Seien \(\sigma\) und \(\theta\) Substitutionen. Für jeden term t gilt \(t(\hat{\sigma\theta}) = (t\hat{\sigma})\theta \) \\ Beweis Strukturelle Induktion über t \(\to\) Übung
\subsubsection{Proposition 4.13}
Sei \(t \in T(F,V) \) und seien \(\sigma, \theta    \text{ sowie } \lambda \) Substitutionen. Dann gilt:
\begin{itemize}
	\item \( t (\hat{(\sigma\theta)\lambda})\)
	\item \(\sigma\theta)\lambda = \sigma(\theta\lambda)\)
\end{itemize}
Beweis siehe Folien (19)
\subsection{Beschränkung von Substitutionen}
\subsubsection{Definition 4.14}
Sei \(\sigma\) eine Substitution. Dann ist 
\begin{displaymath}
	\sigma_{x} = \begin{cases} \sigma &\text{wenn } X \notin \text{dom}( \sigma ) \\
\sigma / \{ X \mapsto t \} &\text{wenn } X \mapsto t \in\sigma
\end{cases}
\end{displaymath}
\subsubsection{Proposition 4.15}
Sei \(\sigma\) eine Substitution und t ein Term, in dem die Variable X nicht vorkommt.\\ Dann gilt: \(t\sigma = t\sigma_{X} \)
\subsection{Anwendung von Substitutionen auf Formeln}
\subsubsection{Definition 4.16}
Die Anwendung einer Substitution \(\sigma\) auf eine Formel ist induktiv über den Aufbau prädikatenlogische Formel wie folgt definiert:
\begin{itemize}
	\item \( p(t_{1} , \ldots , t_{n} ) \sigma = p(t_{1}\sigma , \ldots , t_{n}\sigma ) \)
	\item \((\neg F)\sigma = \neg (F\sigma) \)
	\item \((F\circ G)\sigma = (F\sigma \circ G\sigma)\) für jeden binären Junktor \(\circ\)
	\item \(((QX)F)\sigma = (QX)(F\sigma_{X})\) für jeden Quantor \(Q\)
\end{itemize}
\subsubsection{Beobachtung}
Bei der Anwendung einer Substitution auf eine Formel werden nur frei vorkommende Variablen ersetzt\\
Beweis: Übung
\subsection{Substitutionen und Formeln}
\subsubsection{Definition 4.17}
Eine Substitution \(\sigma\) ist genau dann frei für eine prädikatenlogische Formel F, wenn sie sich gemäß der folgenden bedingungen als frei erweist:
\begin{itemize}
	\item \(\sigma\) ist frei für \(F\), wenn \(F\) ein Atom ist
	\item \(\sigma\) ist frei für \(\neg F\) \textbf{gdw} \(\sigma\) ist frei für \(F\)
	\item \(\sigma\) ist frei für \((F\circ G)\) \textbf{gdw} \(\sigma\) ist frei für \(F\) und \(\sigma\) ist frei für \(G\)
	\item \(\sigma\) ist frei für \((QY)F\) \textbf{gdw} \(\sigma_{Y}\) ist frei für \(F\) und für jede von \(Y\) verschiedene und in \(F\) frei vorkommende Variable \(X\) gilt: \(Y\) kommt in \(X\sigma\) nicht vor 
\end{itemize}
\subsection{Satz 4.18}
\subsubsection{Satz 4.18}
Wenn die Substitution \(\sigma\) frei für die prädikatenlogische Formel \(F\) und die Substitution \(\theta\) frei für \(F\sigma\) ist, dann gilt: \(F(\sigma\theta) = (F\sigma)\theta \)
\subsubsection{Beweis Satz 4.18}
Strukturelle Induktion über \(F\)
\begin{itemize}
	\item \textbf{IA} \(F\) ist Atom der Form \(p(t_{1} , \ldots , t_{n})\)
	\begin{align*}
		&p(t_{1} , \ldots , t_{n} )(\sigma\theta)\\&=p(t_{1}(\sigma\theta) , \ldots , t_{n}(\sigma\theta)) &&\text{Def 4.16}\\ &= p((t_{1}\sigma)\theta , \ldots , (t_{n}\sigma)\theta) &&\text{Prop 4.12}\\ &= p(t_{1}\sigma, \ldots , t_{n}\sigma)\theta &&\text{Def 4.16}\\ &= p(t_{1}, \ldots , t_{n})\sigma)\theta &&\text{Def 4.16}
	\end{align*}
	\item \textbf{IH} Das Resultat gilt für \(F\)
	\item \textbf{IS}
	\begin{itemize}
		\item \textbf{Fall} \(\neg F\)
		\begin{align*}
		&\text{Sei } \sigma \text{ frei für } \neg F \text{ und } \theta \text{ frei für } (\neg F)\sigma \\ &\text{Da } \sigma \text{ frei für } \neg F \text{ ist, ist } \sigma \text{ auch frei für } F \\ &\text{Da } \theta \text{ frei für} (\neg F)\sigma \text{ und } (\neg F)\sigma = \neg (F\sigma) \text{ ist, ist } \theta \text{ auch frei für } F\sigma \\ &((\neg F)\sigma)\theta = (\neg(F\sigma))\theta = \neg ((F\sigma)\theta) =_{(IH)} \neg (F(\sigma\theta)) = (\neg F)\sigma\theta
		\end{align*}
		\item \textbf{Fall} \((F\circ G) \rightsquigarrow \) Übung
		\item \textbf{Fall} \((\forall X)F\)
		\begin{align*}
		&\text{Sei }\sigma \text{ frei für } (\forall X)F \text{ und } \theta \text{ frei für }((\forall X)F)\sigma \\ &\text{Da }\sigma \text{ frei für } (\forall X)F \text{ ist, ist }\sigma_{X} \text{ frei für } F \\&\text{Da }\theta \text{ frei für } ((\forall X)F)\sigma = (\forall X)(F\sigma_{X}) \text{ ist, ist } \theta_{X} \text{ frei für } F\sigma_{X} \\& \textbf{Hilfsaussage } F(\sigma_{X}\theta_{X}) = F(\sigma\theta)_{X} \\&\text{Dann gilt: }
		\end{align*}
		\begin{align*}
		&(((\forall X)F)\sigma)\theta\\&=((\forall X)(F\sigma_{X}))\theta &&\text{Def 4.16} \\&=(\forall X)((F\sigma_{X})\theta_{X}) &&\text{Def 4.16} \\&=(\forall X)(F(\sigma_{X}\theta_{X})) &&\text{IH} \\&=(\forall X)(F(\sigma\theta)_{X})&&\text{Hilfsaussage} \\& = ((\forall X)F)(\sigma\theta) &&\text{Def 4.16}
		\end{align*}
		\item \textbf{Fall} \(\exists X)F \rightsquigarrow\) Übung
	\end{itemize}
\end{itemize}

\subsection{Beweis Hilfsaussage aus Satz 4.18}
Unter den genannten Bedingungen gilt \(F(\sigma_{X}\theta_{X}) = F(\sigma\theta)_{X} \)\\ \textbf{Beweis} Da in \(F\) nur frei vorkommende Variablen ersetzt werden, genügt es zu zeigen, dass für jede frei in \(F\) vorkommende Variable \(Y\) gilt: \(Y(\sigma_{X}\theta_{X}) = Y(\sigma\theta)_{X} \)
\begin{itemize}
	\item \textbf{Fall} \(Y = X\)
	\begin{align*}
		&Y(\sigma_{X}\theta_{X}) = Y = Y(\sigma\theta)_{X}
	\end{align*}
	\item \textbf{Fall} \(Y \ne X\)
	\begin{align*}
	&Y\sigma = Y\sigma_{X} \text{ und } Y(\sigma\theta) = Y(\sigma\theta)_{X}\\&\text{Da }\sigma \text{ frei für } (\forall X)F \text{ ist, kommt die Variable } X \text{ in } Y\sigma \text{ nicht vor}\\&\text{Deshalb ist } (Y\sigma)\theta = (Y\sigma)\theta_{X} \\&\text{Dann gilt:}
	\end{align*}
	\begin{align*}
	&Y(\sigma_{X}\theta_{X})\\&=(Y\sigma_{X})\theta_{X} &&\text{Prop 4.12}\\&=(Y\sigma)\theta_{X} && (X \ne Y)\\&=(Y\sigma)\theta && X \text{ kommt in } Y \text{ nicht vor}\\&= Y(\sigma\theta) &&\text{Prop 4.12}\\&= Y(\sigma\theta)_{X} && (X \ne Y)\\&
	\end{align*}
\end{itemize}

\subsection{Varianten}
\subsubsection{Definition 4.19}
Seien \(E_{1} \) und \(E_{2} \) zwei Terme oder zwei prädikatenlogische Formeln. \(E_{1}\) und \(E_{2}\) heißen Varianten, wenn es Substitutionen \(\sigma \) und \(\theta \) gibt, so dass \(E_{1} = E_{2}\sigma \) und \(E_{2} = E_{1}\theta \). In diesem Fall wollen wir \(E_{1}\) auch als Variante von \(E_{2}\) und \(E_{2}\) als Variante von \(E_{1}\) bezeichnen. \\
Wenn \(E_{1}\) und \(E_{2}\) Varianten sind und die in \(E_{2}\) vorkommenden Variablen im bisherigen Kontext nicht verwendet wurden, dann ist \(E_{2}\) eine neue Variante von \(E_{1}\).

\section{Semantik}
\subsection{Relationen und Funktionen}
\begin{itemize}
	\item Sei D eine Menge ...
	\item Relationen ... 
	\item \textbf{Notation}: Anstelle von \((d)\) schreibt man häufig kurz \(d\)
	\item Funtionen (\(+,\circ,\))
\end{itemize}
\subsection{Interpretationen}
\subsubsection{Definition 4.20}
Eine prädikatenlogische Interpretation \(I\) für eine prädikatenlogische Sprache \(L(R,F,V)\) besteht aus einer nichtleeren Menge \(D\) und einer Abbildung \(\cdot ^I, \) die die folgenden Bedingungen erfüllt:
\begin{itemize}
	\item Jedem n-stelligen Funktionssymbol \(g\in F\) wird eine n-stellige Funktion \(g^I:D^n\to D\) zugeordnet
	\item Jedem n-stelligen Prädikatssymbol \(p\in R\) wird eine n-stellige Relation \(p^I \subseteq D^n\) zugeordnet
\end{itemize}
\(D\) wird Grundbereich oder auch Domäne der Interpretation genannt
\subsubsection{Definition 4.21}
Eine Variablenzuordnung bezüglich einer Interpretation \(I=(D,\cdot ^I)\) ist eine Abbildung \(Z: V\to D\) Das Bild einer Variablen \(X\) unter \(Z\) bezeichnen wir mit \(X^Z\) Sei \(Z\) eine Variablenzuordnung und \(d\in D\),  mit \(\{X \to d\}Z\) bezeichnen wir die Variablenzuordnung für die gilt:
\begin{displaymath}
	Y^{\{ X\mapsto d\}Z}  = \begin{cases}
		d &\text{wenn } Y = X \\
		Y^Z &\text{sonst }
		\end{cases}
\end{displaymath}
\subsubsection{Definition 4.22}
Sei \(I=(D,\cdot ^I)\) eine Interpretation und \(Z\) eine Variablenzuordnung bezüglich \(I\). Die Bedeutung \(t^{I,Z}\) eines Terms \(t\in T(F,V)\) ist wie folgt definiert:
\begin{itemize}
	\item Für jede Variable \(X\in V: X^{I,Z} = X^Z\)
	\item Für jeden Term der Form \(g(t_{1}, \ldots, t_{n})\) ist
	\begin{displaymath}
		[g(t_{1}, \ldots, t_{n})]^{I,Z}=g^I(t_{1}^{I,Z},\ldots,t_n^{I,Z})
	\end{displaymath}
	wobei \(g/n\in F\) ist und \(t_1, \ldots, t_n \in T(F,V)\) sind
\end{itemize}
\subsubsection{Definition 4.23}
Sei \(I=(D,\cdot ^I)\) eine Interpretation und \(Z\) eine Variablenzuordnung bezüglich \(I\) und \(I\) und \(Z\) weisen jeder Formel \(F\in L(R,F,V)\) einen Wahrheitswert \(F^{I,Z} \) wie folgt zu: 
\begin{align*}
&[p(t_1,\ldots ,t_n)]^{I,Z} =  \top \text{ gdw } (t_1^{I,Z},\ldots ,t_n^{I,Z} \in p^I \\ &[\neg F )]^{I,Z} = \neg^*(F^{I,Z}) \\
&[(F\circ G)]^{I,Z} = (F^{I,Z}\circ^* G^{I,Z}) \text{ für alle binären Juntkoren }\circ \\
&[(\forall X)F]^{I,Z}=\top \text{ gdw } F^{I,\{X \mapsto d\}Z} = \top \text{ für alle } d \in D \\
&[(\exists X)F]^{I,Z}=\top \text{ gdw } F^{I,\{X \mapsto d\}Z} = \top \text{ für alle } d \in D \\
\end{align*}
\subsubsection{Proposition 4.24}
Wenn \(F\in L(R,F,V)\) abgeschlossen ist, dann gilt \(F^{I,Z} = F^{I,Z^I}\) für jede Interpretation \(I\) und alle Variablenzuordnungen \(Z\) und \(Z^I\) bezüglich \(I\)
\subsubsection{Lemma 4.25}
Seien \(s,t\) Terme, \(G\) eine Formel, \(Y\) eine Variable, \(I=(D,\cdot ^I)\) eine Interpretation, \(Z\) eine Variablenzuordnung bzgl. \(I\) und \(d\in D\). Wenn \([t]^{I,Z}= d\) ist, dann gilt:
\begin{align*}
&[s\{Y\mapsto t\}]^{I,Z}=[s]^{I,\{Y\mapsto d\} Z} \\
&[G\{Y\mapsto t\}]^{I,Z}=[G]^{I,\{Y\mapsto d\} Z}\text{, wenn } \{Y\mapsto t\} \text{ frei für } g \text{ ist }
\end{align*}
\\\textbf{Beweis} Induktion über den Aufbau von \(s\) bzw. \(G\) \(\rightsquigarrow\) Übung
\subsection{Herbrand-Interpretationen}
Im Folgenden nehmen wir an, dass \(F\) mindestens ein Konstantensymbol enthält. Ist das nicht der Fall, dann fügern wir zu \(F\) ein Symbol \(a/0\) hinzu
\subsubsection{Definition 4.26}
Sei \(F\) eine Menge von Funktionssymbolen, in der mindstens ein Konstantensymbol vorkommt. Eine Interpretation \(I=(D,\cdot ^I)\) für eine prädikatenlogische Sprache \(L(R,F,V)\) ist eine Herbrand-Interpretation, wenn die folgenden Bedingungen erfüllt sind:
\begin{align*}
	&D = T(F) \text{ (D wird Herbrand-Universum genannt})\\
	&\text{Für jeden abgeschlossenen Term } t\in T(F) \text{ gilt } t^I = t
\end{align*}
\section{Modelle}
\subsubsection{Herbrand-Interpretationen und Formeln (41)}
\subsubsection{Aufgabe (42)}
\subsection{Modelle für abgeschlossene Formeln}
\subsubsection{Definition 4.27}
Sei \(I = (D,\cdot ^I)\) eine Interpretation und \(F\in L(R,F,V)\) ein Satz \(I\) ist ein Modell für \(F\), symbolisch \(I \vDash F\), wenn gilt: \(F^I = \top \) \\
Viele aus der Aussagenlogik bekannte Begriffe und REsultate lassen sich auf die Prädikatenlogik übertragen. Zum Beispiel:
\begin{itemize}
	\item Allgemeingültigkeit, Erfüllbarkeit, Widerlegbarkeit und Unerfüllbarkeit
	\item z.B.: Ein Satz \(F\) ist allgemeingültig gdw alle Interpretationen Modelle für \(F\) sind
	\item Satz 3.14 erweitert: Ein Satz \(F\) ist allgemeingültig gdw  \(\neg F\) ist unerfüllbar
	\item Satz 3.17 erweitert: Seien \(F,F_{1}, \ldots, F_{n}\) Sätze\\ \(\{F_{1}, \ldots , F_{n}\} \vDash F\) gdw \(\vDash (\langle F_{1}, \ldots, F_{n} \rangle \to F)\)
\end{itemize}
\subsubsection{Definition 4.28}
Ein Satz F ist eine (prädikatenlogische) Konsequenz der Menge \(G\) von Sätzen, symbolisch \\ \(G \vDash F\), gdw. jedes Modell für alle Elemente aus \(G\) auch Modell für \(F\) ist.
\subsubsection{Aufgaben (45)}
\subsubsection{Aussagenlogik - Prädikatenlogik}
\begin{itemize}
	\item Wenn alle Relationssymbole in \(R\) nullstellig sind, dann ist die Prädikatenlogik äquivalent zur Aussagenlogik
	\item Wenn in der Formeln keine Variablen vorkommen, dann ist die Prädikatenlogik äquivalent zur Aussagenlogik
\end{itemize}
\subsection{Modelle für nicht-abgeschlossene Formeln}
\subsubsection{Definition 4.29}
Sei \(G \in L(R,F,V) \) und \(fv(G) = \{X_{1}, \ldots , X_{n}\}\)
\begin{align*}
& ucl(G) = (\forall (X_{1}, \ldots , X_{n}) G \text{ ist der universelle Abschluss von } G \\
& ecl(G) = (\exists X_{1}, \ldots , X_{n}) G \text{ ist der existenzielle Abschluss von } G
\end{align*}
\subsubsection{Definition 4.30}
\begin{align*}
& I \vDash _{u} \text{ wenn } I \vDash ucl(G) \\
& I \vDash _{e} \text{ wenn } I \vDash ecl(G) 
\end{align*}
\subsubsection{Proposition 4.31}
Für alle Sätze G gilt:
\begin{align*}
& ucl(G) = G = ecl (G) \\
& I \vDash G \text{ gdw } I \vDash _{u} G \text{ gdw } I \vDash_{e} G
\end{align*}
\subsubsection{Universeller Abschluss: Einige Eigenschaften (49)}
\begin{align*}
&\vDash _{u} ((\forall X)p(X) \to p(X)) \\
&\nvDash _{u} (p(X) \to (\forall X)p(X) 
\end{align*}

\section{Äquivalenz und Normalform}
\subsection{Semantische Äquivalenz}
\subsubsection{Definition}
Zwei prädikatenlogische Formeln \(F\) und \(G\) heißen \textbf{semantisch äquivalent}, symbolisch \(F\equiv G\), wenn \(F^{I,Z} = G^{I,Z}\) für alle Interpretationen \(I\) und alle Variablenzuweisungen \(Z\) bezüglich \(I\)
\subsubsection{Satz 3.19}
Für prädikatenlogische Sätze, Formeln gilt:
\begin{align*}
F \equiv G \leftrightarrow F^I = G^I
\end{align*}
\subsubsection{Satz 4.32}
Seien F und G prädikatenlogische Formeln. Dann gilt:
\begin{align*}
\neg(\forall X)F&\equiv (\exists X)\neg F \\
\neg(\exists X)F &\equiv (\forall X)\neg F \\
((\forall X) F \wedge (\forall X)G) &\equiv (\forall X)(F\wedge G)\\
\\
&fill
\end{align*}
\subsubsection{Beweis 1}
siehe Script, die anderen in Eigenarbeit
\subsubsection{Definition 4.33}
Die in einer Prädikatenlogischen Formel \(F\) vorkommenden Variablen sind \textbf{auseinander dividiert}, wenn keine zwei in \(F\) vorkommenden Quantoren die gleiche Variable binden und keine Variable sowohl frei als auch gebunden vorkommt
\subsubsection{Definition 4.33}
Zu jeder prädikatenlogischen Formel gibt es eine semantisch äquivalente Formel, in der die Variablen auseinander dividiert sind
\subsubsection{Vereinbarung}
In der Folge nehmen wir an, dass die Variablen auseinander dividiert sind.
\subsection{Pränexnormalform}
\subsubsection{Definition 4.35} 
Eine Formel \(G\) ist in \textbf{Pränexnormalform}, wenn sie von der Form \((Q_1 X_1)\ldots (Q_n X_n)F \) ist, wobei \(Q_i \in \{\forall,\exists\}\), \(\le i \le n\) und \(n \ge 0 \) ist, \(X_1, \ldots , X_n\) Variablen sind und in \(F\) selbst kein Quantor mehr vorkommt. Wir nennen \(F\) auch Matrix von G
\subsubsection{Proposition 4.36} 
Es gibt einen Algorithmus, der einen Satz \(F\) in der Prädikatenlogik in einen semantisch äquivalenten Satz \(F'\) in Pränexnormalform transformiert
\subsubsection{Transformation in Pränexnormalform}
Solange die Formel \(F\) nciht in Pränexnormalform ist, wende eine der folgenden Regeln an:
\begin{align*}
&\frac{\neg(\forall X)F}{(\exists X)\neg F} 
&& \frac{\neg(\exists X)F}{(\forall X)\neg F} \\
&\frac{((QX)F \wedge G)}{(QX)(F \wedge G)} 
&&\frac{(F\wedge (QX)G)}{(QX)(F\wedge G)} \\
&\frac{((QX)F\lor G)}{(QX)(F\lor G)} 
&&\frac{(F\lor (QX)G)}{(QX)(F\lor G)}
\end{align*}
\subsection{Skolem-Normalform}
\subsubsection{Idee}
Wir beseitigen alle existenziellen Quantoren
\subsubsection{Definition 4.37}
Sei \(L = L(R,F,V)\) eine prädikatenlogische Sprache. Sei \(F_S\) eine abzählbar unendliche Menge von Funktionssymbolen, so dass \(F_S \cap F = \emptyset \) und \(F_S\) für jede Stelligkeit abzählbar unendlich viele Funktionszeichen enthält. Die Elemente aus \(F_S\) werden \textbf{Skolem-Funktionssymbole} genannt. Wir betrachten nun die Sprache \(L(R,F\cup F_S, V)\)
\subsubsection{Notation}
0-stellige Skolem-Funktionssymbole werden häufig auch \textbf{Skolem-Konstantensymbole} genannt
\subsubsection{Definition 4.38}
Eine prädikatenlogische Formel ist in \textbf{Skolem-Normalform}, wenn sie von der Form \(\forall X_	) \ldots (\forall X_n)f\) ist, wobei \(n\geq 0\) ist, \(X_1 ,\ldots ,X_n\) Variablen sind und in \(F\) selbst kein Quantor mehr vorkommt
\subsubsection{Transformation in Skolem-Normalform}
Sei \(F\) Formel in Pränexnormalform, deren variablen auseinander dividert sind. Solange \(F\) nicht in Skolem-Normalform ist, wende die folgende Regel an
\begin{align*}
\frac{(\forall X_1) \ldots (\forall X_n)(\exists Y)G}{(\forall X_1)\ldots (\forall X_n)G \{Y\mapsto f(X_1 ,\ldots , X_n)\} }
\end{align*}
\subsubsection{Satz 4.39}
Sei \(F'\) eine Skolem-Normalform des Satzes \(F\). \(F\) ist erfüllbar gdw  \(F'\) ist erfüllbar.\\ \(\to\) Die Tranformation in Skolem-Normalform ist erfüllbarkeitserhaltend
\subsubsection{Beweis Satz 4.39}
\begin{itemize}
	\item Anname: \(F\) in Pränexnormalform; Variablen sind auseinander dividiert
	\item \textbf{Hilfsaussage}: Sei \(F\) ein Satz in Pränexnormalform, in der die Variablen auseinander dividert sind. Sei \(F'\) durch einmalige Anwendung der Ersetzungsregel auf \(F\) entstanden. Dann gilt: \(F\) ist erfüllbar gdw. \(F'\) ist erfüllbar
	\item Beweis Hilfsaussage \(\rightsquigarrow\) Übung
	\item Sei \(E\) die folgende Zusicherung: \(F'\) ist ein Satz in Pränesnormalform, in der alle Variablen auseinander dividiert sind und \(F'\) ist erfüllbar gdw \(F\) ist erfüllbar
	\item mit \(F = F'\) ist \(E\) vor Eintritt in die Schleife erfüllt
	\item Gemäß der Hilfsaussage ist \(E\) eine Schleifeninvariante
	\item Nach \textbf{Satz 3.30} gilt \(E\) dann auch nach Verlassen der Schleife
	\item Die Schleife wird nur verlassen wenn \(F'\) in Skolem-Normalform ist
\end{itemize}
\subsection{Klauselform}
