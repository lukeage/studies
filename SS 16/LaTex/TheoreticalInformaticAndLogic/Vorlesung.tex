%!TEX root = ../../head.tex

\chapter{Vorlesung}
\section{Prädikatenlogik erster Stufe}

\begin{itemize}
\item Syntax
\begin{itemize}
	\item Ein Alphabet der Prädikatenlogik besteht aus ... (2)
	\item forall heist universeller Quantor, exists heißt							existenzieller Quantor
	\item Funktions- und Relationssymbolen ist eine Stelligkeit n el N			\item Nullstellige Funktionssymbole werden als ... (3)
\end{itemize}
\item Terme
\begin{itemize}
	\item Definition 4.2 prädikatenlogische Terme (4)
	\item Ein Term ist abgeschlossen oder grundinstanziiert, wenn in ihm 			keine Variablen vorkommen
	\item Die Menge der abgeschlossenen Terme wird mit \textit{T}						(\textit{F}) bezeichnet
\end{itemize}
\item Prädikatenlogische Atome (5)
\item Prädikatenlogische Formeln (6)
\begin{itemize}
	\item prädikatenlogische Formeln
\end{itemize}
\item Strukturelle Rekursion 
\begin{itemize}
	\item Rekursionssätze lassen sich für \textit{T}(\textit{F},					\textit{V}) und \textit{L}(\textit{R},\textit{F},\textit{V}) 				formulieren 
	\item Es gibt genau eine Funktion \textit{foo} die die folgenden Bedingungen 				erfüllt: (7)
	\begin{itemize}
		\item Rekursionsanfang
		\item Rekursionsschritt
	\end{itemize}
	\item Beispiele (8/9)
\end{itemize}
\end{itemize}

\section{Prädikatenlogik erster Stufe}
\begin{itemize}
	\item Strukturelle Induktion
	\begin{itemize}
		\item Induktionssätze lassen sich für T(F,V) und L(R,F,V) 						formulieren
		\item jeder Term besitzt die Eigenschaft E, wenn: (10)
		\item analog für prädikatenlogische Formeln
	\end{itemize}
	\item Aufgabe (11)
	\begin{itemize}
		\item Beweisen Sie, dass \begin{math}\forall F \in L(R,F,V) 			\end{math} die Aussage \begin{math} l'(m(F))\ge l(F) \end{math} gilt 
	\end{itemize}
	\item Teilterme und Teilformeln (12)
	\begin{itemize}
		\item Die Def. 3.8 lässt sich auf Terme und Formeln übertragen
		\item Beispiel
	\end{itemize}
	\item Freie und gebundene Vorkommen einer Variablen (13)
	\begin{itemize}
		\item Def. 4.5 Die \textbf{freien Vorkommen einer Variablen} in 				einer prädikatenlogischen Formel sind wie folgt definiert: 					(13)
	\end{itemize}
	\item Abgeschlossene Terme und Formeln (14)
	\begin{itemize}
		\item nach Def. 4.2: Ein abgeschlossener Term ist ein Term, in 						dem	keine Variable vorkommt
		\item Def. 4.6 Eine abgeschlossene Formel (oder kurz 							ein Satz) der Sprache  \textit{L(R,F,V)} ist eine Formel der 			Sprache \textit{L(R,F,V)}, in der jedes Vorkommen einer 					Variablen gebunden ist
	\end{itemize}
	\item Substitutionen (19)
	\begin{itemize}
		\item Def. 4.7: Eine \textbf{Substitution} ist eine Abbildung 					\begin{math} \sigma : V \to T(F,V)\end{math}, die bis auf 					endlich viele Stellen mit der Identitätsabbildung 							übereinstimmt
		\item Beispiel
	\end{itemize}
	\item Instanzen
	\begin{itemize}
		\item Statt \begin{math} \sigma(X) \end{math} schreiben wirn in 				der Folge \begin{math} X\sigma \end{math}
		\item Def. 4.8: Sei sigma eine Substitution \begin{math} \sigma 				: V \to T(F,V)\end{math} kann wie folgt zu einer Abbildung 					\begin{math} \sigma dach: T(F,V) \to T(F,V)\end{math} 						erweitert werden: (25)
		\item Grundinstanz
		\item Proposition
	\end{itemize}
	\item Komposition von Substitutionen
	\begin{itemize}
		\item Def. 4.10: Seien \(\sigma\) und \(\theta\) zwei 							Substitutionen Die 	Komposition \(\sigma\theta\) von 						\(\sigma\) und \(\theta\) ist die Substitution: (30)
		\item Aufgaben
	\end{itemize}
	\item Komposition von Substitutionen (33)
\end{itemize}
