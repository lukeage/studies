%!TEX root = ../../head.tex

\chapter{Übung}
tobias.philipp@tu-dresden.de\\ Fr. 14:00 2005/2006
\section{Syntax}
\subsection{Konstruktion von Teiltermen}
\begin{enumerate}
	\item Bestimme für den Term \( t = h(f(Y,X),Y,g(a)) \) gemäß obiger Definition 2 die Mengen \(K_{n}(t)\) für alle \(n \in \mathbb{N} \)  und geben Sie für alle Terme \(s\in K_{n}(t)\) die zugehörige Konstruktion an.
	\begin{align*}
	&K_{0}(t) = \{t\}&&[t]\\
	&K_{1}(t) = \{f(Y,X),X,g(a)\}&&[t,f(Y,X)],[t,X],[t,g(a)]\\
	&K_{2}(t) = \{Y,X,a\}&&[t,f(Y,X),Y],                          [t,f(Y,X),X],[t,g(a),a]\\
	&K_{3}(t) = \emptyset\\
	&\Rightarrow K(t) = \{t,f(Y,X),X,g(a),Y,a\}
	\end{align*}			

	\item Zeigen Sie, dass \(T_{t} = K(t) \) für beliebige \(t \in T(F,V)\) gilt.
	\begin{align*}
	&\text{z.Z.: }T_{t} = \{t,f(Y,X),X,g(a),Y,a\}\\
	&\text{1. Idee: } T_{t} = K(t) \forall t \in T(F,V)\\
	&\text{2. Idee: Strukt. Induktion}
	\end{align*}
	\item Strukturelle Induktion
	\begin{itemize}
		\item IA
		\begin{align*}
		&\textbf{Fall a) } t \text{ ist von der Form } X\in V \\&\text{Wir wissen, dass } K_{0}(t) = {X} \text{ Außerdem gilt } K_{n}(t) = \emptyset \forall n \ge 1. \\&\text{Aus der Def. von  } K(t) \text{, wissen wir dann dass } K(t) = \{X\} \\&\text{z.Z.: } T_{t} = \{X\} \\& \Rightarrow \{X\} \text{ erfüllt Bed 1 und 2 der Def 1 und Minimalität}\\& \text{1. ist offensichtlich erfüllt} \\&\text{2. ist erfüllt, weil die Vorbedingung immer falsch ist} \\&\text{Minimalität über } \emptyset \\&\Rightarrow K(t) = T_{t} \\&\textbf{Fall b) }t \text{ ist v.d.F. Atom } \in F \textbf{ analog}
		\end{align*}
		\item IV Die Aussage gelte für \(t_{1},\ldots,t_{n})\)
		\item IS z.Z Die Aussage gilt für \(f(t_{1},\ldots,t_{n}) \)
		\begin{align*}
		T_{f(t_{1},\ldots,t_{m})}&=K(f(t_{1},\ldots,t_{m}))\\&=\bigcup_{n = 0}^\infty {K_{n}(f(t_{1},\ldots,t_{m}))}\\&\text{zuerst} \supseteq \\&\text{Sei } s\in \bigcup_{n = 0}^\infty {K_{n}(f(t_{1},\ldots,t_{m}))} \text{ Dann ex. ein }l\text{ sodass } \\& s\in K_{l}(f(t_{1},\ldots,t_{m}) [s_{0},\ldots,s_{l}] \\&\text{Dann gibt es Konst der Länge } l-1 \text{ von } s \text{ aus } t_{i}, i \in \{1,\ldots,m\} \\& s\in K_{l-1}(t_{i}) \forall i \\&\Rightarrow s\in K(t_{i}) \\&\Rightarrow s \in T_{t_{i}} \text{nach I.V.}\\& \text{Es fehlt} \subseteq 
		\end{align*}
	\end{itemize}
\end{enumerate}
\subsection{Über Nachbarn}
\begin{enumerate}
	\item Definieren Sie ...
	\begin{align*}
		&(\forall X)\Big((\exists Y)(m(X)\wedge e(X,Y))\leftrightarrow \text{ vater}(X)\Big)\\
		&(\forall X)(\exists Y)(m(X)\wedge e(X,Y))\leftrightarrow \text{ vater}(X)\\
	\end{align*}
	\item Drücken Sie ...
	\begin{align*}
	&(\forall X)\neg n(X,X)\\
	&(\forall X)\neg (d(X,X) \wedge n(X,X))
	\end{align*}
\end{enumerate}
\section{Substitutionen}
\subsection{Substitutionskomposition ist eine Substitution}
Siehe Lösungen Übungsbuch
\begin{align*}
&\text{z.Z.: } \sigma\theta \text{ ist eine Funktion} \\
&\text{1. Fall: Wenn } X\notin dom(\sigma\theta) \\
&\nexists \text{ Paar } (X\mapsto t\theta ) \in M_1 \text{ mit } X \ne t\theta \\
&\nexists \text{ Paar } (X\mapsto s) \notin M_2 \\
&\text{Dann gibt es ein solches Paar } (X\mapsto s) \notin (M_1 \cup M_2)\\
&\text{ X bildet auf sich selbst ab } \\
&\text{2. Fall: } \\
&\nexists ( Paar ) (X\mapsto t\sigma)\in M_1 , X\ne t\theta \text{, und } \\
&\exists \text{ Paar } (X\mapsto s)\in M_2 \\
& \text{z.Z.: } dom(\sigma\theta) \text{ ist endlich } \\
& \{X|X\in U \text{ und } \sigma\theta (X) \ne X\} \\
& \text{DEF.: }\sigma\theta \sim \{X\mapsto t\theta | X \mapsto t\in \sigma, X \ne t\theta \} \\
& \cup \{ Y \mapsto s | Y \mapsto s \in \theta, Y \notin dom(\sigma) \} \\
& dom(\sigma\theta) \subseteq dom(\sigma ) \cup dom(\theta) \\
&\text{Die Vereinigung von endlichen Mengen ist endlich. } \\ 
&\text{Da } dom(\sigma\theta) \text{ eine Teilmenge ist, folgt Endlichkeit. } \\
&\text{Sei } X\in dom(\sigma\theta)\\
&\text{Dann } \sigma\theta (X) \ne X \\
&\text{Dann muss es ein Paar: }\\
& 1. (X\mapsto t)\in M_{1} \text{ oder }\\
& 2. (X\mapsto t)\in M_{2} \text{ aufgen können}\\
& 1. \exists X\mapsto s \in \sigma , t = s\theta\\
& X \in dom(\sigma), X \in dom(\sigma)\cup dom(\theta) \\
& 2. X\mapsto t \in \theta, X \in dom(\theta), X\in dom(\sigma) \cup dom(\theta)
\end{align*}
\subsection{Eingenschaften von Substitutionen}
Proposition 4.12
\begin{enumerate}
	\item Zeigen Sie, dass für alle Variablen \(X \in V\) gilt: \\
	\( X(\widehat{\sigma\theta}) = (X\hat \sigma )\hat \theta \)
	\begin{align*}
		&\text{1. } \exists X\mapsto t\in \sigma\theta \\
		&\ldots
	\end{align*}
	Fallunterscheidung siehe Lehrbuch
	\item Strukturelle Induktion siehe Lehrbuch
	\begin{itemize}
		\item IA
	\end{itemize}
\end{enumerate}

\section{Semantik}
\subsection{Beispiele zur Interpretationsanwendung}
\begin{enumerate}
	\item
	\begin{enumerate}
		\item \(f(g(f(a,g(a))),a)^{I_1}\)
		\begin{align*}
			\to &f^I((g(f(a,g(a)))^I,(a)^I) \\
			&= f^I(g^I(f^I(a^I,(g(a))^I)), 0)\\
			&= f^I(g^I(f^I(0,s(0))), 0)\\
			&= f^I(g^I(s(0)), 0)\\
			&= s(s(0))
		\end{align*}
		\item \(f(g(a),f(a,a))\) äquiv. zu a)
		\item \([(\forall X)(p(X)\to \neg q(q(X)))]^{I,Z}\)
		\begin{align*}
			&\text{Sei Z eine beliebige Variablenzuordnung bzgl. }I_1 \\
			\to & = \top &&\text{ gdw } \\
			& \forall d \in D : [(p(X)\to \neg q (g(X))]^{I,\{X \mapsto d \} Z} = \top &&\text{ gdw }\\
			& \forall d \in D : [p(X)]^{I,\{X \mapsto d \} Z} \to^* [\neg q (g(X)]^{I,\{X \mapsto d \} Z} = \top &&\text{ gdw}\\
			& X^{I,\{X \mapsto d \} Z} \in p^I \text{ impl } [\neg q (g(X))]^{I,\{X \mapsto d \} Z} = \top &&\text{ gdw}\\
			&\text{Angenommen }d\in D\text{ und es gelte } X^{I,\{X \mapsto d \} Z} \in p^I: \\
			&\text{z.Z.: } [\neg q(g(X))]^{I,\{X \mapsto d \} Z} = \top \\
			&[X]^{I,\{X \mapsto d \} Z} = X^{I,\{X \mapsto d \} Z} = d \\
			&p^I = \{0\}\text{, dann muss } d = 0: \\			
			&[\neg q(g(X))]^{I,\{X \mapsto d \} Z} = \top &&\text{gdw} \\
			&\neg^*[q(g(X))]^{I,\{X \mapsto d \} Z} = \top &&\text{gdw}\\
			&[q(g(X))]^{I,\{X \mapsto d \} Z} = \bot &&\text{gdw}\\
			&[q(X)]^{I,\{X \mapsto d \} Z} \notin q^I \\
			&= g^I(X^{I,\{X \mapsto d \} Z}) \notin q^I \\
			&= s(d) \notin q^I\\
			\Rightarrow &\text{ Offensichtlich falsch, da } q^I = D
		\end{align*}
	\end{enumerate}
	\item
	\begin{enumerate}
		\item Bestimmen Sie den Wert von \(g(f(a,g(b)))\) unter der Interpretation \(I_2\)
		\item Geben Sie einen Term t aus ... an
		\item Bestimmen Sie (Formelauswertung)
	\end{enumerate}
\end{enumerate}
\subsection{Verschiedene Interpretationen einer Formel}
\begin{enumerate}
	\item Interpretation betrachten
	\item Geben Sie eine Interpretation \(I_2\) an unter ...
	\item Zeigen Sie, dass die Formel \(F\) nicht allgemeingültig ist
\end{enumerate}
\subsection{Existenz einer Herbrand-Interpretation}
\begin{align*}
	&I = (D, \circ^I), D = T(F) \text{ ist nicht leer, weil wir voraussetzen, dass minestens }\\ &\text{1 Konstantensymbol in unserer Sprache enthalten ist. Jedem n-stelligen Funktionssymbol}\\ &f \text{ weisen wir folgende Funktion zu:}\\
	&f^I(t_1, \ldots, t_n) \mapsto f(t_1, \ldots, t_n) \\
	&p^I = \emptyset \\
	&\text{z.Z.: }\forall t \in T(F): t^I = t
\end{align*}
Beweis über strukturelle Induktion:
\begin{itemize}
	\item[IA]
	\begin{align*}
		&a \in T(F) \text{ Konstantensymbol} \\
		&\to a^I = a
	\end{align*}
	\item[IH]
	\begin{align*}
		\text{Die Aussage gelte } t_1,\ldots, t_n \in T(F)
	\end{align*}
	\item[IS]
	\begin{align*}
		&\forall f(t_1, \ldots, t_n) \in T(F): \\
		&[f(t_1, \ldots,t_n)]^I \\
		&= f^I(t_1^I,\ldots, t_n^I) \\
		&=^{IH} f^I(t_1, \ldots, t_n) \\
		&=^{Def I} f(t_1,\ldots , t_n)
	\end{align*}
\end{itemize}
\section{Äquivalenz und Normalenform}
fill
\section{Unifikation}
fill
\section{Beweisverfahren}
\subsection{Resolutionsverfahren}
\begin{enumerate}
	\item $<[p(X,Y),q(a,Y)],[\neg p(b,a)],[\neg q(Z,V)]>$
	\begin{align*}
	&[X,Y],q(a,Y)] &&(1)\\
	&[\neg p(b,a)] &&(2)\\
	&[\neg q(Z,V)] &&(3)\\
	&res(1,2) &&(4)\\
	&res(3,4) \to [] &&(5)\\
	\end{align*}
	\item
	\begin{align*}
		&[q(f(a),f(b))] &&(1)\\
		&[\neg p(X,Y),\neg p(f(a),g(X,b)),\neg(X,Z)]&&(2)\\
		&[p(f(X),g(Y,b)),\neg q(Y,f(Y))] &&(3)\\
		&[\neg p(X_1,Y_1),\neg p(f(a),g(X_1,b)),\neg(X_1,Z_1)]&&(2')\\
		&[p(f(X_2),g(Y_2,b)),\neg q(Y_2,f(Y_2))] &&(3')\\
		\text{Faktor. von 2' }& \sigma\{X_1 \mapsto f(a), Y_1 \mapsto g(f(a),b)\}: \\
		&[\neg p(f(a),g(f(a),b),\neg q(f(a),Z_1)] &&(4)\\
		&[\neg p(f(a),g(f(a),b),\neg q(f(a),Z_4)] &&(4')\\
		\text{res(1',4') }&[\neg p(f(a),g(f(a),b))]&&(5)\\
		&[(q(f(a)),f(Y_5)] &&(1'')\\
		\text{res(3',1'') } &\sigma = \{Y_2 \mapsto f(a), Y_5 \mapsto f(a)\}\\
		&[p(f(X_2),g(f(a),b))] &&(6)\\
		& [\neg p(f(a),g(f(a),b))] &&(5')\\
		& p(f(X_6),g(f(a),b))] &&(6')\\
		\text{res(5',6') }& \sigma = \{X_6 \mapsto a\}\\
		&[] &&(7)
	\end{align*}
\end{enumerate}
\subsection{Schrittweiser Resolutionsbeweis}
$(\exists Y)(\forall U)(\neg (\forall U)q(U,Y)\lor q(f(Y),U))$
\begin{itemize}
	\item[1] Negation $\neg F$
	\item[2] Variablen auseinander divideren
	\begin{align*}
		\neg (\exists Y)(\forall U)(\neg (\forall Z)q(Z,Y) \lor q(f(Y),U))
	\end{align*}
	\item[3] Pränex-Normal-Form bilden
	\begin{align*}
		&(\forall Y)\neg (\forall U) F_1\\
		&(\forall Y)(\exists U) \neg F_1\\
		&(\forall Y)(\exists U) \neg(\neg(\forall Z)q(Z,Y) \lor q(F(Y),U))\\
		&(\forall Y)(\exists U) \neg ((\exists Z)\neg q(Z,Y) \lor q(f(Y),U)) \\
		&(\forall Y)(\exists U)\neg(\exists Z)(\neg q(Z,Y) \lor q(f(Y),U)) \\
		&(\forall Y)(\exists U)(\forall Z)\neg(\neg q(Z,Y)\lor q(f(Y),U))\\
	\end{align*}
	\item[4] Skolem-Normal-Form bilden
	\begin{align*}
	&(\forall Y) \sigma = \{U\mapsto g(Y)\}\\
	&\rightsquigarrow (\forall Y)(\forall Z)\neg (\neg q(Z,Y) \lor q(f(Y),g(Y)))
	\end{align*}
	\item[5] Klauselform bilden
	\begin{align*}
	&\forall <[\neg(\neg q(Z,Y) \lor q(f(Y),g(Y))]>\\
	&<[ \neg\neg q(Z,Y)],[\neg q(f(Y,g(Y))]>\\
	&<[q(Z,Y)],[\neg q(f(Y),g(Y))]>\\
	\end{align*}
	\item[6] Resolutionsverfahren
	\begin{align*}
	&[q(Z_1,Y_1)] &&(1')\\
	&[\neg q(f(Y_2),g(Y_2))] &&(2')\\
	\text{res(1',2') }& [] && (3)\\
	\end{align*}
\end{itemize}
\subsection{Notwendigkeit der Faktorisierung}
\begin{align*}
	&F = < [p(), p()], [\neg p(), \neg p()]>\\
	\text{res 1,2) }& [p(), \neg p()] &&(3) \\
	\text{fakt 1' }& [p] &&(4)\\
	\text{fakt 2' }& [\neg p] &&(5)\\
	\text{res 4',5') }& [] &&(6)\\
\end{align*}
Beweis über Induktion
\section{Eigenschaften}
\subsection{Beispiel für korrespondierendes Herbrand-Modell}
\begin{align*}
	&D^I = \mathbb{N}\\
	&a^I = 0\\
	&s^I = x \mapsto x + 1\\
	&p^I = \{(x,y,z)|x \le y + z\}\\
\end{align*}
Herbrand Modell
\begin{align*}
	D^J &= T(F) \text{ Menge von Termen über F}\\
	&= \{a, s(a), s(s(a)), \ldots \} \\
	p^J &= \{(s^x(a), s^y(a), s^z(a))|x \le y + z\}
\end{align*}
\section{Nachtrag Logik}
\begin{align*}
	F \vDash p(t) \text{ für alle t}\in T(F) \to F \vDash (\forall X)p(X) ?
\end{align*}
Nein. Hier: häufige Fehler!! Gegenbeispiel:
\begin{align*}
	&L(R,F,V) \text{ bestimmen mit} F = \{c_1, \ldots, c_n\}\\
	&\text{Sei G Formel:}\\
	&G = p(c_1) \wedge p(c_2) \wedge \ldots \wedge p(c_n)\\
	&\text{Wiederspruch ("Missmatch" mit Term und Domäne) mit:}\\
	&D = \{1, \ldots , n+1\}\\
	&c_1^I = \{1, \ldots, c_n^I = n\}\\
	&p^I = \{1, \ldots, n\}
\end{align*}

\section{Turing-Maschine}
\subsection{Palindrom Turingmaschine}
\begin{align*}
	A = &(Q,\{a,b\},\{a,b,\not b, \},q_0,\Delta,F) \\
	\Delta = &\{(q_0,a,\not b, r, q_2) &&(q_0, b, \not b, r, q_3)\\
	&(q_2, \not b, \not b, n, q_4) &&(q_3, \not b, not b, n, q_4)\\
	&(q_2, *, *, r, q_a) &&(q_2, *, *, r, q_b)\\
	&(q_a, *, *, r, q_a) &&(q_b, *, *, r, q_b)\\
	&(q_a, \not b, \not b, l, q_{av}) &&(q_b, \not b, \not b, l, q_{b,v})\\
	&(q_{av}, a, \not b, l, q_{l1}) &&(q_{bv}, b, \not b, l, q_{l1})\\
	&(q_l, x,x, l, q_l) &&x \in \{a,b\}\\
	&(q_{l1}, \not b, \not b, n, q_F) && (q_l, x,x,l,q_l)\\
	&(q_l, \not b, \not b, r, q_0) &&\}\\
\end{align*}
\subsection{Wort 2-Band Turingmaschine}
\begin{align*}
q_0: &\underline{a}ba &&\underline{\not b}\not b \not b\\
q_0: &a\underline{b}a &&a\underline{\not b}\not b\\
q_0: &ab\underline{a} &&ab\underline{\not b}\\
q_0: &aba\underline{\not b} &&aba\underline{\not b}\\
q_1: &aba\underline{\not b} && ab\underline{a}\\
q_1: &aba\underline{\not b} &&a\underline{b}a\\
q_1: &aba\underline{\not b} &&\underline{a}ba\\
q_1: &aba\underline{\not b} &&\underline{\not b}aba\\
q_2: &aba\underline{\not b} && \underline{a}ba\\
q_2: &abaa\underline{\not b} &&\not{b}\underline{b}a\\
q_2: &abaaba\underline{\not b} &&\not{b}\not b\not b\underline{\not b}
\end{align*}

\section{}
\subsection{Berechenbarkeit, Entscheidbarkeit, Aufzählbarkeit}
\begin{enumerate}
	\item wahr, nach Def.
	\item wahr, gilt nach Satz 2.4
	\item wahr, Beweis durch Gegenannahme falsch
	\item falsch: sonst partiell entscheidbar = entscheidbar
\end{enumerate}
\subsection{Primitiv rekursive Funktionen}
\begin{enumerate}
	\item 
	\begin{itemize}
		\item PROD:
		\begin{align*}
			&f: \mathbb{N}^{n+1} \mapsto \mathbb{N} \\
			&g: \mathbb{N}^{n}\mapsto \mathbb{N}\\
			&h: \mathbb{N}^{n+1} \mapsto \mathbb{N}\\
			&\text{falls gilt:}\\
			&f(x_1, \ldots, x_n, 0) = g(x_1, \ldots, x_n)\\
			&f(x_1, \ldots, x_n, y+1) = h(x_1, \ldots, x_n, f(x_1, \ldots, x_n, y), y)\\
			PROD(X,0) &= 0 = g(X) = null^{(1)}(X)\\
			PROD(X, Y+1) &= PROD(X,Y) + Y = h(X, PROD(X,Y),Y)\\
			&= add(\pi_2^{(3)}(X,PROD(X,Y),Y),\pi_3^{(3)}(X,PROD(X,Y),Y)\\
			&= add(\pi_2^{(3)}(X,PROD(X,Y),Y),\pi_3^{(3)}(X,PROD(X,Y),Y))
		\end{align*}
		\item MODDIFF:
		\begin{align*}
	         MD(X,0) &= X = g(X) = id_{(1)}^{(1)}\\
	         MD(X,Y+1) &= MD(X,Y)-1\\
	         min1(X) &= \begin{cases} x-1 &x>0\\
	         0 &\text{sonst} \end{cases}\\
	         &\text{oder:}\\
	         vor(0) &= 0 = null^{(0)}\\
	         vor(X+1)&=X= h(vor(X), X)\\
	         &\text{dann:}\\
	         MD(X,Y+1) &= vor(\pi_2^3(X, MD(X,Y),Y))\\   
		\end{align*}
	\end{itemize}
	\item 
	\begin{itemize}
		\item $f_1$
		\begin{align*} 
		sum(0) &= \Sigma_{i=0}^{0} = 0 = g = null^{(0)} \\
	         sum(y+1) &= \Sigma_{i=0}^{y+1} = add(sum(Y), Y+1)\\
	         &= h(sum(Y),Y)\\
	         &= add(\pi_1^{(2)}(sum(Y),Y),s(\pi_2^{(2)}(sum(Y),Y)))\\
	         f_1(x_1,\ldots,x_k, x_{k+1}) &= sum(g(x_1, \ldots, x_k, i))\\
	         f_1(x_1, \ldots, x_k, 0) &= g(x_1, \ldots, x_k, 0)\\
	         f_1(x_1, \ldots, x_k, Y+1) &= add(f_1(x_1, \ldots, x_k, y), g(x_1, \ldots, x_k, Y+1)) \\
	         &= h(x_1, \ldots, x_k, f_1(x_1, \ldots, x_k, Y), Y)\\
	         &= add\Big(\pi_{k+1}^{(k+2)}\underbrace{\big(x_1, \ldots, x_k, f_1(x_1, \ldots, x_k, Y),Y\big)}_{=l}, \\&g\big(\pi_1^{(k+2)}(l), \ldots, \pi_k^{(k+2)}(l),s(\pi_{k+2}^{(k+2)}(l))\big)\Big)\\
			tadaaaaa!!!
		\end{align*}
		\item $f_2$ sind wir nichmehr zu gekommen
	\end{itemize}
\end{enumerate}