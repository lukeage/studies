\documentclass[a4paper, 12pt] {article} %selbsterklärend
\usepackage[latin1]{inputenc} %lateinischer Schriftsatz (umlaut.)
\usepackage[ngerman]{babel} %das richtige Sprachpaket (deutsche Rechtschreibung)
\usepackage[utf8]{inputenx} % ein anderes richtiges Sprachpaket
\usepackage{a4wide} % sonst lässt er links und rechts je 3 Meilen Rand
\usepackage[T1]{fontenc} %keine Ahnung, aber hatten die bei LLP auch
\usepackage{enumerate} %wichtig für Aufzählungen
\usepackage{xcolor} % buntes Zeugs
\usepackage{soul} % unterstreichen mit Zeilenumbruch
\usepackage{amsmath} %megawichtig für Mathe
\usepackage{nccmath} % ach, das habe ich?
\usepackage{amssymb} % "American Mathematical Society"
\usepackage{stmaryrd} % kein Plan
\usepackage{fancyhdr} % für "fancy" Kopf-Und Fußzeilen
\usepackage{enumitem} % alternatives, schwierigeres Aufzählungszeug
\usepackage{amsthm}
\renewcommand{\arraystretch}{1.3} %damit sehen Tabellen wesentlich besser aus
\parindent0cm % für Anti-Einzug
%\usepackage{hyperref} %Alternative zu titleref 
\usepackage{titleref}


% INF ZEUG
\usepackage{listings} \lstset{numbers=left, numberstyle=\tiny, numbersep=5pt} \lstset{language=c} 


%Document
\title{Rechnernetze 01 Einführung}
\begin{document}
\maketitle

\begin{itemize}
	\item Anwendungsfelder Rechnernetze (1.4)
	\begin{itemize}
		\item Geschäftsanwendungen - gemeinsame Nutzung von 							Resourcen
		\item Privatbereich - Informationszugriff (z.B. WWW, IM)
		\item Mobile Benutzer - Textnachrichten, ...
		\item Gesellschaftliche Aspekte - Copyright, Profile, ...
	\end{itemize}
	\item Client Server Modell (1.5)
	\item Peer-to-Peer Communication (1.6)
	\item Basis-Netzstruktur (1.7)
	\begin{itemize}
		\item Übertragungsmodi
		\begin{itemize}	
			\item Verbindungsorientiert
			\item Verbindungslos (z.B. IP)
			\item Leitungsvermittelt
			\item Paketvermittelt (flexibler, 												ressourcenschonend)
		\end{itemize}
	\end{itemize}
	\item Schichtenarchitektur - ISO/OSI Referenzmodell (1.8)
	\begin{itemize}
		\item International Organization for Standardization
		\item Open Systems Interconnection
		\item Schichtenübersicht auf 1.8 ff.
	\end{itemize}
	\item Integriertes Referenzmodell (Tanenbaum) (1.11)
	\begin{itemize}
		\item Protokollimplementierung oft abweichend vom Referenzmodell
	\end{itemize}
	\item Besipiel Datenübertragung (1.12)
	\item Schichteneffizienz (1.13)
	\item Dienste - Begriffsklärung (1.14)
	\begin{itemize}
		\item Beispiel Ablaufdiagramm (1.15)
	\end{itemize}
	\item Netzkopplung - Basis-Topologien
	\begin{itemize}
		\item Punkt-zu-Punkt-Kanäle (Unicast)
		\item Rundsendekanäle (Broadcast)
		\item Klassifizierung nach Ausdehnung (1.17)
		\begin{itemize}
			\item Pan - Personal Area Network
			\item LAN - Local Area Network
			\item MAN - Metropolitan Aria Network
			\item WAN - Wide Area Network (1.18)
		\end{itemize}
		\item Mobilität || Leistung (1.19)
		\item Konzepte - Layer-N-Gateway(1.20)
		\item Beispiel (1.21)
	\end{itemize}
	\item Internet(1.22 ff)
	\begin{itemize}
		\item Internet
		\begin{itemize}
			\item Geschichte des Internet (1.24 ff)
			\item Normen (1.26)
		\end{itemize}
		\item Intranet (1.22)
	\end{itemize}
	
\end{itemize}




- THE END  :) - 

\begin{lstlisting}

\end{lstlisting}



\end{document}
