% !TEX root = Rechnernetze_All.tex

\part{Rechnernetze}
\section{Einführung}
timo.schick@tu-dresden.de

\subsection{}
\begin{enumerate}
	\item Sterntopologie: Ein zentrales Element(Sternkoppler), jeder 				Rechner benötigt eine Leitung zu Sternkoppler \(\to\) 5
	\item Jeder mit Jedem \(= 4+3+2+1=10\)
	\item
	\begin{enumerate}
		\item \(l(n)= n\) bei Sterntopologie
		\item \(l(n) = \sum ... = (n*(n-1))/2\) bei vollvermaschter 					Topologie
	\end{enumerate}
	\item
	\begin{enumerate}
		\item LAN
		\begin{itemize}
			\item Reichweite: ~10m
			\item Reaktionszeit: niedrig
			\item Datenrate: hoch
			\item Topologien: Sterntopologie
		\end{itemize}
		\item MAN
		\begin{itemize}
			\item Reichweite: ~10km
			\item Reaktionszeit: mittel
			\item Datenrate: mittel
			\item Topologien: hierarchische Topologie
		\end{itemize}
		\item WAN
		\begin{itemize}
			\item Reichweite: ~100km - ~10.000km
			\item Reaktionszeit: hoch
			\item Datenrate: niedrig
			\item Topologien: Vollvermaschte Topologie
		\end{itemize}
	\end{enumerate}
\end{enumerate}

\subsection{}
\begin{enumerate}
	\item Dienst und Protokoll
	\begin{itemize}
		\item siehe Musterlösung
	\end{itemize}
	\item OSI Schichtenmodell
	\begin{itemize}
		\item Schichtenmodell siehe Folie 1.8ff
		\item Protokoll:
		\begin{itemize}
			\item ist eine Sprache zur horizontalen Kommunikation 							zwischen Prozessen derselben Schicht auf verschiedenen 						Hosts
		\end{itemize}
		\item Dienst
		\begin{itemize}
			\item dient der vertikalen Kommunikation zwischen zwei 							Schichten auf einem Host
		\end{itemize}
		\item Aufteilung des Bitstroms: Schicht 2 Sicherungsschicht
		\item Ende-zu-Ende Kommunkation: Schicht 4 Transportschicht
		\item Wegewahl: Schicht 3 Vermittlungsschicht
	\end{itemize}
	\item keine inhaltliche Bearbeitung, sondern nur 								Informationsweiterleitung
\end{enumerate}

\subsection{}
\begin{enumerate}
	\item
	\begin{itemize}
		\item siehe Folie 1.15; 
		\item Initiator (Prozess A), ...
		\item Responder (Prozess B), ...
	\end{itemize}
	\item
	\begin{enumerate}
		\item Zustände bestimmen
		\begin{itemize}
			\item idle
			\item connected
			\item prepare(Initiator)
			\item prepare(Responder)
		\end{itemize}
		\item Übergänge bestimmen (Knoten, Pfad, Knoten)
		\begin{itemize}
			\item (idle, conReq, prep(Init))
			\item (idle, ConInd, prep(Resp))
			\item (prep(Resp), conRsp, connected)
			\item (prep(Init), conCnf, connected)
			\item (connected, dataRep/dataInd, connected) 
			\item (prep(Resp)/prep(Init)/connected, disRep/disInd, idle)
		\end{itemize}
	\end{enumerate}
	\item
	\begin{enumerate}
		\item Ablaufdiagramm
		\begin{itemize}
			\item c1) + zeitlicher Ablauf
			\item c2) - es werden n Diagramme benötigt
			\item c3) - 
		\end{itemize}
		\item Zustandsdiagramm
		\begin{itemize}
			\item c1) - 
			\item c2) + alle Abläufe in einem Diagramm darstellbar
			\item c3) + 
		\end{itemize}
	\end{enumerate}
\end{enumerate}

\subsection{}
\begin{enumerate}
	\item siehe Folie 1.10
	\begin{enumerate}
		\item \(PDU(N)=SDU(N-1)\)
		\item \(IDU(N)=ICI(N)+SDU(N)\)
	\end{enumerate}
	\item Seitenaufruf: http://www.heise.de/software
	\begin{enumerate}
		\item httpRequest
		\begin{enumerate}
			\item  GET/software/http/1.1
			\item Host: www.heise.de
		\end{enumerate}
		\item ICI 
		\begin{enumerate}
			\item ip: 193.99.144.85 port:80
		\end{enumerate}
		\item SDU
		\begin{enumerate}
			\item  GET/software/http/1.1
			\item Host: www.heise.de
		\end{enumerate}
		\item IDU
		\begin{enumerate}
			\item ICI
			\item SDU
		\end{enumerate}
		\item TCP-PDU
		\begin{enumerate}
			\item src:80, dest:80,...
			\item SDU
			\item Data
		\end{enumerate}
	\end{enumerate}
	\item	
	\begin{align*}		
		b_{0}&=125\frac{\text{Mbit}}{\text{s}}\\
		b_{1}&=b_{0}\cdot0,8\\
		b_{2}&=b_{1}\frac{(55+99)0,01}2\\
		b_{3}&=b_{2}\frac{(57+99)0,01}2\\
		b_{4}&=b_{3}\frac{(23+99)0,01}2=36,4\frac{\text{Mbit}}							{\text{s}}\\
		b_{4}&=b_{goodput}\\
		b_{extra}&=b_{2}\frac{(23+99)0,01}2=46,7\frac{\text{Mbit}}						{\text{s}}
	\end{align*}
\end{enumerate}





- THE END  :) - 

\begin{lstlisting}

\end{lstlisting}



\end{document}
