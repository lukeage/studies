\documentclass[a4paper, 12pt] {article} %selbsterklärend
\usepackage[latin1]{inputenc} %lateinischer Schriftsatz (umlaut.)
\usepackage[ngerman]{babel} %das richtige Sprachpaket (deutsche Rechtschreibung)
\usepackage[utf8]{inputenx} % ein anderes richtiges Sprachpaket
\usepackage{a4wide} % sonst lässt er links und rechts je 3 Meilen Rand
\usepackage[T1]{fontenc} %keine Ahnung, aber hatten die bei LLP auch
\usepackage{enumerate} %wichtig für Aufzählungen
\usepackage{xcolor} % buntes Zeugs
\usepackage{soul} % unterstreichen mit Zeilenumbruch
\usepackage{amsmath} %megawichtig für Mathe
\usepackage{nccmath} % ach, das habe ich?
\usepackage{amssymb} % "American Mathematical Society"
\usepackage{stmaryrd} % kein Plan
\usepackage{fancyhdr} % für "fancy" Kopf-Und Fußzeilen
\usepackage{enumitem} % alternatives, schwierigeres Aufzählungszeug
\usepackage{amsthm}
\renewcommand{\arraystretch}{1.3} %damit sehen Tabellen wesentlich besser aus
\parindent0cm % für Anti-Einzug
%\usepackage{hyperref} %Alternative zu titleref 
\usepackage{titleref}


% INF ZEUG
\usepackage{listings} \lstset{numbers=left, numberstyle=\tiny, numbersep=5pt} \lstset{language=c} 


%Document
\title{01 Ü Einführung}
\begin{document}
\maketitle
\author{Luke}

\begin{itemize}
\item 1.1
\begin{itemize}
	\item a)
	\begin{itemize}
		\item Sterntopologie: Ein zentrales Element(Sternkoppler), jeder 			Rechner benötigt eine Leitung zu Sternkoppler \(\to\) 5
	\end{itemize}
	\item b) Jeder mit Jedem \(= 4+3+2+1=10\)
	\item c) 
	\begin{itemize}
		\item \(l(n)= n\) bei Sterntopologie
		\item \(l(n) = \sum ... = (n*(n-1))/2\) bei vollvermaschter 					Topologie
	\end{itemize}
	\item d)
	\begin{itemize}
		\item LAN
		\begin{itemize}
			\item Reichweite: ~10m
			\item Reaktionszeit: niedrig
			\item Datenrate: hoch
			\item Topologien: Sterntopologie
		\end{itemize}
		\item MAN
		\begin{itemize}
			\item Reichweite: ~10km
			\item Reaktionszeit: mittel
			\item Datenrate: mittel
			\item Topologien: hierarchische Topologie
		\end{itemize}
		\item WAN
		\begin{itemize}
			\item Reichweite: ~100km - ~10.000km
			\item Reaktionszeit: hoch
			\item Datenrate: niedrig
			\item Topologien: Vollvermaschte Topologie
		\end{itemize}
	\end{itemize}
\end{itemize}
\item 1.2
\begin{itemize}
	\item a) Dienst und Protokoll
	\begin{itemize}
		\item siehe Musterlösung
	\end{itemize}
	\item b) OSI Schichtenmodell
	\begin{itemize}
		\item Schichtenmodell siehe Folie 1.8ff
		\item Protokoll:
		\begin{itemize}
			\item ist eine Sprache zur horizontalen Kommunikation 							zwischen Prozessen derselben Schicht auf verschiedenen 						Hosts
		\end{itemize}
		\item Dienst
		\begin{itemize}
			\item dient der vertikalen Kommunikation zwischen zwei 							Schichten auf einem Host
		\end{itemize}
		\item Aufteilung des Bitstroms: Schicht 2 Sicherungsschicht
		\item Ende-zu-Ende Kommunkation: Schicht 4 Transportschicht
		\item Wegewahl: Schicht 3 Vermittlungsschicht
	\end{itemize}
	\item c)
	\begin{itemize}
		\item keine inhaltliche Bearbeitung, sondern nur 								Informationsweiterleitung
	\end{itemize}
\end{itemize}
\item 1.3
\begin{itemize}
	\item a)
	\begin{itemize}
		\item siehe Folie 1.15; 
		\item Initiator (Prozess A), ...
		\item Responder (Prozess B), ...
	\end{itemize}
	\item b)
	\begin{itemize}
		\item Zustände bestimmen
		\begin{itemize}
			\item idle
			\item connected
			\item prepare(Initiator)
			\item prepare(Responder)
		\end{itemize}
		\item Übergänge bestimmen (Knoten, Pfad, Knoten)
		\begin{itemize}
			\item (idle, conReq, prep(Init))
			\item (idle, ConInd, prep(Resp))
			\item (prep(Resp), conRsp, connected)
			\item (prep(Init), conCnf, connected)
			\item (connected, dataRep/dataInd, connected) 
			\item (prep(Resp)/prep(Init)/connected, disRep/disInd, idle)
			\end{itemize}
	\end{itemize}
	\item c)
	\begin{itemize}
		\item Ablaufdiagramm
		\begin{itemize}
			\item c1) + zeitlicher Ablauf
			\item c2) - es werden n Diagramme benötigt
			\item c3) - 
		\end{itemize}
		\item Zustandsdiagramm
		\begin{itemize}
			\item c1) - 
			\item c2) + alle Abläufe in einem Diagramm darstellbar
			\item c3) + 
		\end{itemize}
	\end{itemize}
\end{itemize}
\item 1.4
\begin{itemize}
	\item a) siehe Folie 1.10
	\begin{itemize}
		\item \(PDU(N)=SDU(N-1)\)
		\item \(IDU(N)=ICI(N)+SDU(N)\)
	\end{itemize}
	\item b) Seitenaufruf: http://www.heise.de/software
	\begin{itemize}
		\item httpRequest
		\begin{itemize}
			\item  GET/software/http/1.1
			\item Host: www.heise.de
		\end{itemize}
		\item ICI 
		\begin{itemize}
			\item ip: 193.99.144.85 port:80
		\end{itemize}
		\item SDU
		\begin{itemize}
			\item  GET/software/http/1.1
			\item Host: www.heise.de
		\end{itemize}
		\item IDU
		\begin{itemize}
			\item ICI
			\item SDU
		\end{itemize}
		\item TCP-PDU
		\begin{itemize}
			\item src:80, dest:80,...
			\item SDU
			\item Data
		\end{itemize}
	\end{itemize}
	\item c)
	\begin{itemize}
		\item \(b_{0}=125\frac{Mbit}s\)
		\item \(b_{1}=b_{0}*0,8\)
		\item \(b_{2}=b_{1}*\frac{(55+99)*0,01}2\)
		\item \(b_{3}=b_{2}*\frac{(57+99)*0,01}2\)
		\item \(b_{4}=b_{3}*\frac{(23+99)*0,01}2=36,4\frac{Mbit}s\)
		\item \(b_{4}=b_{goodput}\)
		\item \(b_{extra}=b_{2}*\frac{(23+99)*0,01}2=46,7\frac{Mbit}s\)
	\end{itemize}
\end{itemize}
\item timo.schick@tu-dresden.de
\end{itemize}



- THE END  :) - 

\begin{lstlisting}

\end{lstlisting}



\end{document}
