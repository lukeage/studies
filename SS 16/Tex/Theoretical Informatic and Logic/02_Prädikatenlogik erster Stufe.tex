\documentclass[a4paper, 12pt] {article} %selbsterklärend
\usepackage[latin1]{inputenc} %lateinischer Schriftsatz (umlaut.)
\usepackage[ngerman]{babel} %das richtige Sprachpaket (deutsche Rechtschreibung)
\usepackage[utf8]{inputenx} % ein anderes richtiges Sprachpaket
\usepackage{a4wide} % sonst lässt er links und rechts je 3 Meilen Rand
\usepackage[T1]{fontenc} %keine Ahnung, aber hatten die bei LLP auch
\usepackage{enumerate} %wichtig für Aufzählungen
\usepackage{xcolor} % buntes Zeugs
\usepackage{soul} % unterstreichen mit Zeilenumbruch
\usepackage{amsmath} %megawichtig für Mathe
\usepackage{nccmath} % ach, das habe ich?
\usepackage{amssymb} % "American Mathematical Society"
\usepackage{stmaryrd} % kein Plan
\usepackage{fancyhdr} % für "fancy" Kopf-Und Fußzeilen
\usepackage{enumitem} % alternatives, schwierigeres Aufzählungszeug
\usepackage{amsthm}
\renewcommand{\arraystretch}{1.3} %damit sehen Tabellen wesentlich besser aus
\parindent0cm % für Anti-Einzug
%\usepackage{hyperref} %Alternative zu titleref 
\usepackage{titleref}


% INF ZEUG
\usepackage{listings} \lstset{numbers=left, numberstyle=\tiny, numbersep=5pt} \lstset{language=c} 


%Document
\title{02 Prädikatenlogik erster Stufe}
\begin{document}
\maketitle


\begin{itemize}
\item Syntax
\begin{itemize}
	\item Ein Alphabet der Prädikatenlogik besteht aus ... (2)
	\item forall heist universeller Quantor, exists heißt							existenzieller Quantor
	\item Funktions- und Relationssymbolen ist eine Stelligkeit n el N			\item Nullstellige Funktionssymbole werden als ... (3)
\end{itemize}
\item Terme
\begin{itemize}
	\item Definition 4.2 prädikatenlogische Terme (4)
	\item Ein Term ist abgeschlossen oder grundinstanziiert, wenn in ihm 			keine Variablen vorkommen
	\item Die Menge der abgeschlossenen Terme wird mit \textit{T}						(\textit{F}) bezeichnet
\end{itemize}
\item Prädikatenlogische Atome (5)
\item Prädikatenlogische Formeln (6)
\begin{itemize}
	\item prädikatenlogische Formeln
\end{itemize}
\item Strukturelle Rekursion 
\begin{itemize}
	\item Rekursionssätze lassen sich für \textit{T}(\textit{F},					\textit{V}) und \textit{L}(\textit{R},\textit{F},\textit{V}) 				formulieren 
	\item Es gibt genau eine Funktion \textit{foo} die die folgenden Bedingungen 				erfüllt: (7)
	\begin{itemize}
		\item Rekursionsanfang
		\item Rekursionsschritt
	\end{itemize}
	\item Beispiele (8)
	
\end{itemize}

\end{itemize}



- THE END  :) - 

\begin{lstlisting}

\end{lstlisting}



\end{document}
