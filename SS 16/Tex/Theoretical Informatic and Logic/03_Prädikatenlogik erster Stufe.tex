\documentclass[a4paper, 12pt] {article} %selbsterklärend
\usepackage[latin1]{inputenc} %lateinischer Schriftsatz (umlaut.)
\usepackage[ngerman]{babel} %das richtige Sprachpaket (deutsche Rechtschreibung)
\usepackage[utf8]{inputenx} % ein anderes richtiges Sprachpaket
\usepackage{a4wide} % sonst lässt er links und rechts je 3 Meilen Rand
\usepackage[T1]{fontenc} %keine Ahnung, aber hatten die bei LLP auch
\usepackage{enumerate} %wichtig für Aufzählungen
\usepackage{xcolor} % buntes Zeugs
\usepackage{soul} % unterstreichen mit Zeilenumbruch
\usepackage{amsmath} %megawichtig für Mathe
\usepackage{nccmath} % ach, das habe ich?
\usepackage{amssymb} % "American Mathematical Society"
\usepackage{stmaryrd} % kein Plan
\usepackage{fancyhdr} % für "fancy" Kopf-Und Fußzeilen
\usepackage{enumitem} % alternatives, schwierigeres Aufzählungszeug
\usepackage{amsthm}
\renewcommand{\arraystretch}{1.3} %damit sehen Tabellen wesentlich besser aus
\parindent0cm % für Anti-Einzug
%\usepackage{hyperref} %Alternative zu titleref 
\usepackage{titleref}


% INF ZEUG
\usepackage{listings} \lstset{numbers=left, numberstyle=\tiny, numbersep=5pt} \lstset{language=c} 


%Document
\title{03 Prädikatenlogik erster Stufe}
\begin{document}
\maketitle
\author{Luke}


\begin{itemize}
	\item Strukturelle Induktion
	\begin{itemize}
		\item Induktionssätze lassen sich für T(F,V) und L(R,F,V) 						formulieren
		\item jeder Term besitzt die Eigenschaft E, wenn: (10)
		\item analog für prädikatenlogische Formeln
	\end{itemize}
	\item Aufgabe (11)
	\begin{itemize}
		\item Beweisen Sie, dass \begin{math}\forall F \in L(R,F,V) 			\end{math} die Aussage \begin{math} l'(m(F))\ge l(F) \end{math} gilt 
	\end{itemize}
	\item Teilterme und Teilformeln (12)
	\begin{itemize}
		\item Die Def. 3.8 lässt sich auf Terme und Formeln übertragen
		\item Beispiel
	\end{itemize}
	\item Freie und gebundene Vorkommen einer Variablen (13)
	\begin{itemize}
		\item Def. 4.5 Die \textbf{freien Vorkommen einer Variablen} in 				einer prädikatenlogischen Formel sind wie folgt definiert: 					(13)
	\end{itemize}
	\item Abgeschlossene Terme und Formeln (14)
	\begin{itemize}
		\item nach Def. 4.2: Ein abgeschlossener Term ist ein Term, in 						dem	keine Variable vorkommt
		\item Def. 4.6 Eine \textbf{abgeschlossene} Formel (oder kurz 					ein Satz) der Sprache  \textit{L(R,F,V)} ist eine Formel der 			Sprache \textit{L(R,F,V)}, in der jedes Vorkommen einer 					Variablen gebunden ist.
	\end{itemize}
	\item Substitutionen (19)
	\begin{itemize}
		\item Def. 4.7: Eine \textbf{Substitution} ist eine Abbildung 					\begin{math} \sigma : V \to T(F,V)\end{math}, die bis auf 					endlich viele Stellen mit der Identitätsabbildung 							übereinstimmt
		\item Beispiel
	\end{itemize}
	\item Instanzen
	\begin{itemize}
		\item Statt \begin{math} \sigma(X) \end{math} schreiben wirn in 				der Folge \begin{math} X\sigma \end{math}
		\item Def. 4.8: Sei sigma eine Substitution \begin{math} \sigma 				: V \to T(F,V)\end{math} kann wie folgt zu einer Abbildung 					\begin{math} \sigma dach: T(F,V) \to T(F,V)\end{math} 						erweitert werden: (25)
		\item Grundinstanz
		\item Proposition
	\end{itemize}
	\item Komposition von Substitutionen
	\begin{itemize}
		\item Def. 4.10: Seien \(\sigma\) und \(\theta\) zwei 							Substitutionen Die 	Komposition \(\sigma\theta\) von 						\(\sigma\) und \(\theta\) ist die Substitution: (30)
		\item Aufgaben
	\end{itemize}
	\item Komposition von Substitutionen (33)
\end{itemize}



- THE END  :) - 

\begin{lstlisting}

\end{lstlisting}



\end{document}
